\section{Lemma 2.9.}

% https://q.uiver.app/#q=WzAsMyxbMCwwLCJcXG1hdGhjYWwgRihWX3gpIl0sWzIsMCwiXFxtYXRoY2FsIEYoVV94KSJdLFsxLDEsInNfeFxcaW5cXG1hdGhjYWwgRl94Il0sWzAsMSwiXFx0ZXh0e1Jlc30iXSxbMCwyXSxbMSwyXV0=
\[\begin{tikzcd}
	{\mathcal F(V_x)} && {\mathcal F(U_x)} \\
	& {s_x\in\mathcal F_x}
	\arrow["{\text{Res}}", from=1-1, to=1-3]
	\arrow[from=1-1, to=2-2]
	\arrow[from=1-3, to=2-2]
\end{tikzcd}\]

Assume $s_x=0$, while in Category of Abelian groups the homomorphism is exactly group homomorphism. So there must exist some $U_x$ such that $s\mid_{U_x}=0$ as a pre-image of $s_x$.

\section{Exercise 2.7.}
\textit{Let $\mathcal B$ be a base of open subsets on a topological space $X$. Let $\mathcal F,\mathcal G$ be two sheaves on $X$. Suppose that for every $U\in\mathcal B$ there exists a homomorphism $\alpha(U):\mathcal F(U)\to\mathcal G(U)$ which is compatible with restrictions. Show that this extends in a unique way to a homomorphism of sheaves $\alpha:\mathcal F\to\mathcal G$. Show that if $\alpha(U)$ is surjective (resp. injective) for every $U\in\mathcal B$, then $\alpha$ is surjective (resp. injective). }

\begin{proof}
	There are two definitions of sheaf on the whole space in terms of sheaf on base. One is to define section of an open subset $W\subset X$ (open subset $W$ is not necessarily in base $\mathcal B$) by \textit{compatible germs}, another approach is to define that by \textit{limit}. 

	To construct the map, we just take the definition of \textit{limit}. And the universal property will induce a unique map $\alpha(W)$ for arbitrary open subset $W\subset X$, which implies there exists a unique morphism $\alpha:\mathcal F\to\mathcal G$ as we expected. 

	Notice that for sheaves on base, the stalk at a point $x\in X$ is the same as the sheaf on whole space. 
	\begin{align*} 
		\underset{x\in W\subset X,~ W \text{ open}}{\operatorname{colim}}=\mathcal F_x=\underset{x\in U\subset X,~ U\in\mathcal B}{\operatorname{colim}}.
	\end{align*} Another way to see this (compatible germs) is by 2.5.B. EXERCISE. \cite{RaviRisingSea}. Therefore we have the following diagram, which maps to the same stalk.

	% https://q.uiver.app/#q=WzAsOCxbMCwwLCJcXG1hdGhjYWwgRihVKSJdLFsxLDAsIlxcbWF0aGNhbCBHKFUpIl0sWzAsMSwiXFxtYXRoY2FsIEYoVVxcY2FwIFVfeCkiXSxbMSwxLCJcXG1hdGhjYWwgRyhVXFxjYXAgVV94KVxcbmkgdFxcdmVydF97VVxcY2FwIFVfeH0iXSxbMCwyLCJzXFxpblxcbWF0aGNhbCBGKFcpIl0sWzEsMiwiXFxtYXRoY2FsIEcoVylcXG5pIHRcXHZlcnRfVyJdLFsxLDMsIlxcbWF0aGNhbCBHX3hcXG5pIHRfeCJdLFswLDMsIihXLHMpXFxpblxcbWF0aGNhbCBGX3giXSxbMCwxLCIiLDAseyJzdHlsZSI6eyJoZWFkIjp7Im5hbWUiOiJlcGkifX19XSxbNCw1LCIiLDAseyJzdHlsZSI6eyJoZWFkIjp7Im5hbWUiOiJlcGkifX19XSxbMiwzXSxbMCwyXSxbMiw0XSxbMSwzXSxbMyw1XSxbNCw3XSxbNSw2XSxbNyw2XV0=
\[\begin{tikzcd}
	{\mathcal F(U)} & {\mathcal G(U)} \\
	{\mathcal F(U\cap U_x)} & {\mathcal G(U\cap U_x)\ni t\vert_{U\cap U_x}} \\
	{s\in\mathcal F(W)} & {\mathcal G(W)\ni t\vert_W} \\
	{(W,s)\in\mathcal F_x} & {\mathcal G_x\ni t_x}
	\arrow[two heads, from=1-1, to=1-2]
	\arrow[two heads, from=3-1, to=3-2]
	\arrow[from=2-1, to=2-2]
	\arrow[from=1-1, to=2-1]
	\arrow[from=2-1, to=3-1]
	\arrow[from=1-2, to=2-2]
	\arrow[from=2-2, to=3-2]
	\arrow[from=3-1, to=4-1]
	\arrow[from=3-2, to=4-2]
	\arrow[from=4-1, to=4-2]
\end{tikzcd}\]

	For surjective. We wish to prove the induced map on stalks $\alpha_x:\mathcal F_x\to\mathcal G_x$ is surjective for any $x\in X$. While $\mathcal B$ covers the whole space, we can pick some $U\in\mathcal B$ such that $x\in U$. Given any $t_x\in\mathcal G_x$, i.e. we have $t\in\mathcal G(U_x)$ for some open subset $x\in U_x\subset X$. In case that $U\cap U_x$ isn't in the base, we can further restrict to $t\vert_{W}$ for some $W\in\mathcal B$. By commutativity of the diagram we know there's some $s\in \mathcal F(W)$ such that $s_x=(W,s)$ will be mapped to $t_x$. Hence the map induced on stalk $\alpha_x:\mathcal F_x\to\mathcal G_x$ is surjective. 

	% https://q.uiver.app/#q=WzAsNixbMCwwLCJcXG1hdGhjYWwgRihVKSJdLFsxLDAsIlxcbWF0aGNhbCBHKFUpIl0sWzAsMSwiXFxtYXRoY2FsIEYoVykiXSxbMSwxLCJcXG1hdGhjYWwgRyhXKSJdLFswLDIsIlxcbWF0aGNhbCBGX3giXSxbMSwyLCJcXG1hdGhjYWwgR194Il0sWzAsMV0sWzIsMywiIiwyLHsic3R5bGUiOnsidGFpbCI6eyJuYW1lIjoiaG9vayIsInNpZGUiOiJ0b3AifX19XSxbMCwyXSxbMSwzXSxbNCw1XSxbMiw0XSxbMyw1XV0=
\[\begin{tikzcd}
	{\mathcal F(U)} & {\mathcal G(U)} \\
	{\mathcal F(W)} & {\mathcal G(W)} \\
	{\mathcal F_x} & {\mathcal G_x}
	\arrow[from=1-1, to=1-2]
	\arrow[hook, from=2-1, to=2-2]
	\arrow[from=1-1, to=2-1]
	\arrow[from=1-2, to=2-2]
	\arrow[from=3-1, to=3-2]
	\arrow[from=2-1, to=3-1]
	\arrow[from=2-2, to=3-2]
\end{tikzcd}\]

	For injective. Now we start by picking two elements $a_x,b_x\in\mathcal F_x$, where $a\in \mathcal F(U_a)$ and $b\in\mathcal F(U_b)$. Then we pick an open subset $W\in\mathcal B$ such that $B\subset U\cap U_a\cap U_b$. The $\alpha(W)$ is injective by assumption. Then $a\vert_{W}$ and $b\vert_W$ will be mapped to the same element in $\mathcal G_x$. By commutativity this proves the injective of $\alpha_x$.

\end{proof}

\section{Proposition 3.1.}

\subsection{(b)}

For the proof of isomorphism, it implicitly used Corollary 7.5 of \cite{atiyah1994introduction}.

\section{Lemma 3.35.}

\subsection{(a)}

Take $n$ to be the largest index such that $a_n\notin I^h$, and choose $b_m$ similarly. They must exist for $a\notin I^h$ by assumption.
And in general, if we have $a\notin I^h$, we can always subtract all homogeneous part in the tail of $a$ to get a new $a'$. And denote $a'=a$.