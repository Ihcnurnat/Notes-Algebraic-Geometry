\section{2.2.6. Definition: Sheaf.}

Comments on $\mathscr{F} (\emptyset)$. In category $\textbf{Set}$, the empty set is initial object and one element set is terminal. See Wiki's examples \href{https://en.wikipedia.org/wiki/Initial_and_terminal_objects#:~:text=Similarly%2C%20the%20empty%20space%20is,hence%20the%20unique%20zero%20object.}{HERE}.

\section{2.2.B.}

For (a): see Wiki's counterexample \href{https://en.wikipedia.org/wiki/Sheaf_(mathematics)}{HERE}, which gave an explanation for presheaves on $\mathbb R$ instead of $\mathbb C$.
See a post \href{https://math.stackexchange.com/questions/2946197/why-is-the-bounded-functions-not-a-sheaf}{HERE}.

\section{2.2.7.}

See Daping's Notes Definition 2.5 on Page 4 \href{http://www.math.uchicago.edu/%7Emay/VIGRE/VIGRE2011/REUPapers/WengD.pdf}{HERE}; also a post \href{https://math.stackexchange.com/questions/4080618/describe-sheaf-properties-via-equalizers}{HERE}; also a post \href{https://math.stackexchange.com/questions/453203/definition-of-sheaf-using-equalizer}{HERE}.

\section{2.2.10.}
It's different from a post \href{https://math.stackexchange.com/questions/195363/constant-presheaf-not-necessarily-a-sheaf-proof}{HERE}, and Wiki's page on \href{https://en.wikipedia.org/wiki/Constant_sheaf}{Constant pre-Sheaf}. \todo{Why???}

\section{2.2.G.}
It's clearly a pre-sheaf.

Fix an open subset $U\subset X$ with an open cover $\{U_i\}_{i\in I}$ for some index set $I$. Denote the presheaf as $\mathscr F$.

Pick two continuous maps $s_1,s_2:Y\to X$ that satisfying the requirements, i.e. $s_1,s_2\in\mathscr F(U)$.

Both functions will agree on $U$ since 
\[\operatorname{Res}_{U,U_i}~ s_1=\operatorname{Res}_{U,U_i}~ s_1\] for arbitrary $U_i$, whose union is $U$. So we must have $s_1=s_2$.

Again with this open cover $\{U_i\}_{i\in I}$ and $a_i\in \mathscr F(U_i)$ for $i\in I$. Equivalently, we know $a_i:U_i\to Y$ is a continuous map satisfying $\mu\circ a_i=\operatorname{Id}\mid_{U_i}$.
Now let's define a map 
\begin{align*}
    f:U &\to Y\\
    u &\mapsto a_i(u) ~\text{ when }~ u\in U_i.     
\end{align*}It's well-defined by our assumption. Also it's continuous since preimage of an open set in $V\subset Y$ is a union of open subsets given by continuity of each $a_i$. Similarly we can check $\mu\circ f=\operatorname{Id}\mid_{U}$ as expected.\todo{Unverified ?}

\section{2.2.11. Espace \'Etal\'e}
See a post discussion accent letter in LaTeX \href{https://tex.stackexchange.com/questions/8857/how-to-type-special-accented-letters-in-latex}{HERE}.

See an exercise in \cite{hartshorne2013algebraic} Chapter 2, Exercise 1.13.

See the discussion after Lemma 7. on Page 229 of \cite{bosch2013algebraic}.

For \textit{section}, see Wiki's explanation for \href{https://en.wikipedia.org/wiki/Section_(fiber_bundle)}{\textit{section}} in context of fiber bundle; and \href{https://en.wikipedia.org/wiki/Section_(category_theory)}{\textit{section}} in terms of category theory.

\section{2.3.A.}

I'm planning to use universal property to define the induced map $\phi_P$.

One crutial step is to verify the diagram below is commutative

% https://q.uiver.app/#q=WzAsNSxbMCwwLCJcXG1hdGhzY3IgRihVKSJdLFswLDEsIlxcbWF0aHNjciBHKFUpIl0sWzIsMCwiXFxtYXRoc2NyIEYoVikiXSxbMiwxLCJcXG1hdGhzY3IgRyhWKSJdLFsxLDIsIlxcbWF0aHNjciBHX1AiXSxbMCwxLCJcXHBoaShVKSIsMl0sWzAsMiwiXFxyaG9fe1VWfSJdLFsxLDMsIlxcdGF1X3tVVn0iLDJdLFsyLDMsIlxccGhpKFYpIl0sWzEsNF0sWzMsNF1d
\[\begin{tikzcd}
	{\mathscr F(U)} && {\mathscr F(V)} \\
	{\mathscr G(U)} && {\mathscr G(V)} \\
	& {\mathscr G_P}
	\arrow["{\phi(U)}"', from=1-1, to=2-1]
	\arrow["{\rho_{UV}}", from=1-1, to=1-3]
	\arrow["{\tau_{UV}}"', from=2-1, to=2-3]
	\arrow["{\phi(V)}", from=1-3, to=2-3]
	\arrow[from=2-1, to=3-2]
	\arrow[from=2-3, to=3-2]
\end{tikzcd}\]

And this is because the square diagram in the upper half commute given $\phi$ is a natural transformation; the lower half is by definition of $\mathscr G_P$. Then by universal property of colimit induces a map
\[\phi_P:\mathscr F_P\to\mathscr G_P\] which makes the diagram commute.

See a post defined the map \href{https://math.stackexchange.com/questions/1387214/the-induced-map-on-stalks-is-well-defined}{HERE}.

\section{2.3.B.}
To define a functor $\pi_{\ast}:\textbf{Set}_X\to\textbf{Set}_Y$.
Firstly, we have to define for any $\mathscr F\in \textbf{Set}_X$, 
\[\pi_{\ast}(\mathscr F)(U)=\mathscr F(\pi^{-1}(U))\] for any $U\in \mathfrak{Top}(X)$ as in \ref{2.2.H.}. 

Secondly, for any natural transformation $\phi:\mathscr F\to\mathscr G$, we define $\pi_{\ast}(\phi)$ by specifying 
\[\pi_{\ast}(\phi)(U) ~\mapsto \mathscr F(\pi^{-1}(U))\to\mathscr G(\pi^{-1}(U)).\]\todo{? Is this correct}

