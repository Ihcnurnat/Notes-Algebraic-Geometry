\section{3.1.A.}

?

\section{3.2.A.}

\subsection{Example 5}

A complete description of $\operatorname{Spec}(\mathbb R[x])$ is given by a post \href{https://math.stackexchange.com/questions/1057642/describe-the-topology-of-spec-mathbbrx}{HERE}.

To find all prime ideals in P.I.D. $\mathbb R[x]$ is generated by one single element, i.e. a polynomial $f\in \mathbb R[x]$. While $\mathbb R[x]$ is U.F.D., a prime element is equivalent to an irreducible element. Therefore we need to find all irreducible polynomials in $\mathbb R[x]$. Notice that field extension 
\[[\mathbb C:\mathbb R]=2 ~\Rightarrow~ \operatorname{deg} f\leq 2\] for any irreducible polynomial $f$. 

In case of $\operatorname{deg} f=2$, it's precisely the case where we require $f$ to be an irreducible quadratic. 

See a post \href{https://math.stackexchange.com/questions/2127249/what-are-the-irreducible-elements-in-mathbbrx-and-mathbbcx}{HERE}.

\section{3.2.B.}

For an irreducible polynomial $x^2+ax+b\in\mathbb R[x]$, it cannot admit a real root. Therefore in algebraic closure of $R$, i.e. in $\mathbb C$, we can find two roots $\alpha_1,\alpha_2$ of the polynomial. 

\section{3.2.C.}

?

\section{3.2.D.}

I didn't figure out the Euclid's proof for this. 

For Euclid's proof, see a post \href{https://math.stackexchange.com/questions/80389/ring-of-polynomials-over-a-field-has-infinitely-many-primes}{HERE}.

Basic ideal is consider 
\[f=\prod_{i\in I}f_i+1\]
where $\langle f_i\rangle=\mathfrak p_i$ for a \textit{finite} index set $I$. Here $\mathfrak p_i\triangleleft k[x]$ is a prime ideal. Each $f_i$ must have degree $\operatorname{deg}f_i\geq 1$ because otherwise $\mathfrak p_i$ will contain a unit. While in integral domain $k[x]$, we know degree $f$ is again large or equal to $1$, hence not a unit. Clearly it's nonzero. As a non-zero non-unit element in U.F.D. $k[x]$, it could be written as a product of irreducible elements, say $g_1$. It is will also generate a prime ideal, which means \[\langle g_1\rangle =\mathfrak p_i=\langle f_i\rangle ~\Rightarrow~ g_1=uf_i\] for some $i\in I$ and unit $u$. Then $f\equiv 1\pmod{g_i}$, which means $g_i$ will divide unit $1$, contradiction.

See a post \href{https://math.stackexchange.com/questions/585201/can-operatornamespecrx-ever-be-finite}{HERE}, \href{https://math.stackexchange.com/questions/496202/operatornamespeckx-has-infinite-points}{HERE}.

\section{3.2.L. Exercise}

\begin{proof}
    Localisation commute with quotient, therefore we plan to prove the isomophism by constructing a ring homomorphism from $\psi:\mathbb C[x,y]_x\to \mathbb C[x]_x$ and compute the kernel. 
    The map $\psi$ is defined as 
    \begin{align*}
        \psi:\mathbb C[x,y]_x &\to \mathbb C[x]_x\\
            f(x,y)/x^i &\mapsto f(x,0)/x^i
    \end{align*}where $f(x,y)\in\mathbb C[x,y]$ and $i$ is some integer. 
    
    Clearly we see $(xy)_x\subset\operatorname{Ker}\psi$. Conversely, let's pick an element $f(x,y)/x^i$ such that $f(x,0)/x^i=0\in \mathbb C[x]_x$, this implies there exists $j\in\mathbb N$ such that 
    \[x^jf(x,0)=0\in\mathbb C[x] ~\Rightarrow~ f(x,0)=0\in\mathbb C[x].\] \todo{stopped,...}

    The conclusion follows immediately if we realise 
    \[(y)_x=(xy)_x\subset \mathbb C[x,y]_x.\]
    One way to interpret this is as follow: both $(x)_x$ and $(xy)_x$ are image of localisation map of a principal ideas $(x),(xy)\triangleleft\mathbb C[x,y]$. And two principal ideals are the same if they differ by a unit, say $x$. Or we can perform a double inclusion argument.
\end{proof}

\subsection{Hint}

See a post \href{https://math.stackexchange.com/questions/3769097/showing-that-mathbbcx-y-xy-x-cong-mathbbcx-x}{HERE}.

Some details are not clear, so see the following.
\begin{proof}
    Two facts to recall is that we have two exact sequences 
    % https://q.uiver.app/#q=WzAsMTAsWzAsMCwiMCJdLFsxLDAsIih4eSlfeFxcc3Vic2V0IFxcbWF0aGJiIENbeCx5XV94Il0sWzIsMCwiXFxtYXRoYmIgQ1t4LHldX3giXSxbMywwLCIoXFxtYXRoYmIgQ1t4LHldLyh4eSkpX3giXSxbNCwwLCIwIl0sWzAsMSwiMCJdLFs0LDEsIjAiXSxbMSwxLCIoeSlcXHN1YnNldCBcXG1hdGhiYiBDW3gseV0iXSxbMiwxLCJcXG1hdGhiYiBDW3gseV0iXSxbMywxLCJcXG1hdGhiYiBDW3hdIl0sWzAsMV0sWzEsMl0sWzIsM10sWzMsNF0sWzUsN10sWzksNl0sWzgsOV0sWzcsOF1d
\[\begin{tikzcd}
	0 & {(xy)_x\subset \mathbb C[x,y]_x} & {\mathbb C[x,y]_x} & {(\mathbb C[x,y]/(xy))_x} & 0 \\
	0 & {(y)\subset \mathbb C[x,y]} & {\mathbb C[x,y]} & {\mathbb C[x]} & 0
	\arrow[from=1-1, to=1-2]
	\arrow[from=1-2, to=1-3]
	\arrow[from=1-3, to=1-4]
	\arrow[from=1-4, to=1-5]
	\arrow[from=2-1, to=2-2]
	\arrow[from=2-4, to=2-5]
	\arrow[from=2-3, to=2-4]
	\arrow[from=2-2, to=2-3]
\end{tikzcd}\]
Here the second rows will induce, by exactness of localisation 
% https://q.uiver.app/#q=WzAsNSxbMCwwLCIwIl0sWzQsMCwiMCJdLFsxLDAsIih5KV94PSh4eSlfeFxcc3Vic2V0IFxcbWF0aGJiIENbeCx5XV94Il0sWzIsMCwiXFxtYXRoYmIgQ1t4LHldX3giXSxbMywwLCJcXG1hdGhiYiBDW3hdX3giXSxbMCwyXSxbNCwxXSxbMyw0XSxbMiwzXV0=
\[\begin{tikzcd}
	0 & {(y)_x=(xy)_x\subset \mathbb C[x,y]_x} & {\mathbb C[x,y]_x} & {\mathbb C[x]_x} & 0
	\arrow[from=1-1, to=1-2]
	\arrow[from=1-4, to=1-5]
	\arrow[from=1-3, to=1-4]
	\arrow[from=1-2, to=1-3]
\end{tikzcd}\]
And this concludes the desired isomorphism. One crutial observation I missed was $(xy)_x=(y)_x$.

\end{proof}