\section{11.1.A.}

According to the diagram, notice that 
\begin{align*}
    X\times_Y Y \times_Y Y\times_Z Y &\simeq X\times_Z Y,\\
    X\times_Z Y \times_X Z &\simeq Z\times_Z Y.
\end{align*}

\section{11.1.E. Exercise}

First of all, structure morphism must automatically lies in $P$ because $P$ contains all isomorphisms, therefore applying $S\to S$ base change gives us $S\times_S A=A\to S$ has property $P$. 
\[% https://tikzcd.yichuanshen.de/#N4Igdg9gJgpgziAXAbVABwnAlgFyxMJZABgBpiBdUkANwEMAbAVxiRAA0QBfU9TXfIRQBGUsKq1GLNgGVuvEBmx4CRAEzkJ9Zq0QgAmgB1DeALbwA+jIAEALW4SYUAObwioAGYAnCKaRkQHAgkUUkdNmMsMGsABRBqBjoAIxgGGP4VIRAvLGcACxx5Tx8-RA1A4MRQ7Wk9SOiY41gGHDoikG9ff2ogpHKa3RBjHDyYLxgPCHHrAD962IcuIA
\begin{tikzcd}
X \arrow[rd, "\in P"'] \arrow[rr, "\therefore ~\in P"] &   & Y\times_S Z \arrow[ld, "\in P\delta"] \\
                                                       & S &                                      
\end{tikzcd}\] 

\section{11.2.C.}

Use Exercise 1.2.W, we have the diagonal morphism is an isomorphism. Hence closed embedding.

\section{11.2.E.}

Assume $X$ is separated over $\Spec A$, $\Spec A\to \Spec \mathbb Z$ is an affine morphism, we can compose two separated morphism to conclude $X$ is separated (over $\Spec \mathbb Z$). 

Conversely, consider the following diagram: 
\[% https://tikzcd.yichuanshen.de/#N4Igdg9gJgpgziAXAbVABwnAlgFyxMJZABgBpiBdUkANwEMAbAVxiRAA0QBfU9TXfIRQAmclVqMWbADrSIaGACc6OCIrB0AtjGABlBQGMuAQW68QGbHgJEAjKVvj6zVohCz5SlWo3a9hrllNFQALACMwgAIALW5xGCgAc3giUAAzRQhNJDIQVSR7CRcZaRwQpRg0tRhIgAUzdMzsxFz8xFEiqTdZLDA6kGoGOjCYBlr+ayEQRSxEkJwGkAysguo2jucu92leutlYBhw6OK4gA
\begin{tikzcd}
X \arrow[rr, "\therefore P"] \arrow[rd, "\in P"'] &                              & \operatorname{Spec}A \arrow[ld, "\in P\delta"] \\
                                                  & \operatorname{Spec}\mathbb Z &                                               
\end{tikzcd}\]

It suffices to check $\Spec A\to \Spec A\times_{\Spec \mathbb Z} \Spec A$ is separated. But this is already an affine morphism so we're done.

\section{11.2.F.}

\subsection{(a)}

\begin{proof}[locally closed embedding]
    For locally closed embedding, we apply Cancellation Theorem for property $P$ of being a \enquote{local closed embedding}. With the existing assumption, it suffices to check $\rho$ is $P\delta$. Recall $Y\to Y\times_Z Y$ is a local closed embedding by 11.2.1. Proposition, hence we're done.
\end{proof}

\begin{proof}[locally of finite type]
    It boils down to check $\rho$ is $P\delta$ for property $P$ of being a \enquote{locally of finite type} morphism. Based on previous part, we know $Y\to Y\times_Z Y$ is a local closed embedding. Hence it's locally of finite type by 9.2.A.
\end{proof}

\begin{proof}[separated]

\end{proof}
Failed to give a proof. See \href{https://stacks.math.columbia.edu/tag/01KV}{Tag 01KV}. 

\section{11.2.G.}
    11.2.F. with Cancellation Theorem for property $P$ of either locally closed embedding or separated morphism can solve the first two parts.
\begin{proof} 
    
\end{proof}

\section{11.3.A.}
\href{https://stacks.math.columbia.edu/tag/01KM}{Tag 01KM}
\section{11.3.C.}

Let $\pi:W=\mathbb A^1\to X$ to be the inclusion of one copy of affine line $\mathbb A^1$, and $\pi'$ to be the inclusion from another copy. They agree on $\mathbb A^1\setminus \{0\}$, which is NOT closed. According to 11.3.A., if $X$ is separated, then two morphisms should agree on a closed subscheme, contradiction.

\section{11.3.2.}

See \href{https://stacks.math.columbia.edu/tag/01RR}{Tag 01RR} and 
\href{https://math.stackexchange.com/questions/2128060/proof-of-the-reduced-to-separated-theorem}{Math Stack Exchange discussion}.