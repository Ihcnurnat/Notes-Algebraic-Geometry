\section{5.1.C.}

The hint in the prompt was refering to Exercise 2.13. \Cref{Hart Chap 2 Ex 2.13.} Chapter 2 of Hartshorne \cite{hartshorne2013algebraic}.

\section{5.1.D.}\label{5.1.D.}

Remarks: $\Rightarrow$ is easy because affine opens form a basis, while the converse direction is non-trivial. The point is that we only know a specific choice of affine opens. 

Fact: we can build a larger quasi-compact topological spaces by union finitely many quasi-compact topological spaces. See this \href{https://math.stackexchange.com/questions/816938/quasicompact-scheme-are-finite-union-of-affine-scheme}{POST}.

\section{5.1.F.}\label{5.1.F.}

$\Rightarrow$: Let $U,V$ be two affine opens in scheme $X$. We know $U,V$ are quasi-compact, then their intersection $U\cap V$ is quasi-compact. While affine opens form a basis for topology of $X$, by quasi-compactness of $U\cap V$ we can cover it by finitely many affine opens. 

$\Leftarrow$: Let $A,B$ be two quasi-compact open subsets of $X$. By quasi-compactness we can decompose them as 
\[A=\bigcup_{i=1}^n A_i,~ B=\bigcup_{j=1}^m B_j.\] where all $A_i,B_j$ are affine opens. 
Notice that \[A\cap B=\bigcup_{1\leq i\leq n,~ 1\leq j\leq m}(A_i\cap B_j)\] is a finite union of $A_i\cap B_j$, by assumption it's a finite union of affine opens. Apply \Cref{5.1.D.} we conclude $A\cap B$ is quasi-compact, which proves $X$ is quasi-separated.

\section{5.1.G.}\label{5.1.G.}

Zariski topology admits distinguished open as a basis. For solution see this \href{https://math.stackexchange.com/questions/130502/affine-schemes-are-quasi-separated}{POST}.

\section{5.1.H.}\label{5.1.H.}

$\Rightarrow$: If we take \Cref{5.1.F.} as an equivalent definition of quasi-separated, then we can take the open affine cover as the original affine opens. 

$\Leftarrow$: ??? 

\subsection{Hint}

See \href{https://math.stackexchange.com/questions/4625360/scheme-quasiseparated-iff-covered-by-affine-open-subsets-any-two-of-which-have}{POST} discussing this exercise. 

\section{5.2.A.}

\begin{proof}
    
See 3.2.1. When we say two functions $f,g$ agree at all points, this means 
\[f\equiv g\pmod{p}\] for all $p\in \Gamma(X,\mathcal O_X)$.    

To prove $f=g$, we wish to use the Identity Axiom of Sheaf. Fix an open covering $\{U_i\}_{i\in I}$ of $X$ where $I$ an index set. It suffices to check $f\mid_{U_i}=g\mid_{U_i}$.

We can identify them by injective map 
% https://q.uiver.app/#q=WzAsMixbMCwwLCJcXG1hdGhjYWwgT19YKFgpIl0sWzEsMCwiXFxwcm9kX3twXFxpbiBYfVxcbWF0aGNhbCBPX3tYLHB9Il0sWzAsMSwiIiwwLHsic3R5bGUiOnsidGFpbCI6eyJuYW1lIjoiaG9vayIsInNpZGUiOiJ0b3AifX19XV0=
\[\begin{tikzcd}
	{\mathcal O_X(U_i)} & {\prod_{p\in U_i}\mathcal O_{X,p}}
	\arrow[hook, from=1-1, to=1-2]
\end{tikzcd}\]
This forces $f=g$ when they have same ???

\end{proof}

\subsection{References}

See this \href{https://math.stackexchange.com/questions/3834821/equality-of-functions-in-a-reduced-scheme}{POST}: but is the definition of value of function correct?

See another \href{https://math.stackexchange.com/questions/1157904/functions-on-reduced-schemes-are-determined-by-their-values-at-each-point}{POST}.

\section{5.2.C.}

Localisation of reduced ring is again reduced for reduce is a local property. 

For $\mathbb A^n_k$: 
polynomial ring $k[x_1,...,x_n]$ is reduced.

For $\mathbb P^n_k$: it's similar, since taking subring will preserve reducedness.

\section{5.3.1. Proposition}

\section{}
We'll try to interpret the equality in the proof by applying Hartshorne \cite{hartshorne2013algebraic} Chap II Exercise 2.16.
\begin{align*}
	D_B(g)=&~ \operatorname{Spec}B_g\\
	=&~ \operatorname{Spec}B_g\cap \operatorname{Spec}A_f\\
	=&~ (\operatorname{Spec}B)_g \cap \operatorname{Spec}A_f ~\text{ by Hart Ex 2.16}\\
	=&~ D_{A_f}(g') ~\text{ by Hart Ex 2.16}\\
	=&~ (\operatorname{Spec}A_f)_{g'}\\
	=&~ \operatorname{Spec}A_{fg''}.
\end{align*}
Here $g''$ is specified as in the textbook \cite{RaviRisingSea}, and $g'$ is defined as \[g\in B_g\to g'\in\Gamma(\operatorname{Spec}A_f,\mathcal O_X\vert_{\operatorname{Spec}A_f})=A_f.\]

For $\operatorname{Spec} B_g=\operatorname{Spec}(A_f)_{g'}$, we need \cite{hartshorne2013algebraic} Hartshorne Chap II Exercise 2.16. 

\begin{comment}
See this \href{https://youtu.be/EWnhf0BdAWA?si=zMjtGhr12HSCjzNr}{VIDEO}, this \href{https://math.stackexchange.com/questions/384137/intersection-of-open-affines-can-be-covered-by-open-sets-distinguished-in-both}{POST}.

See \href{http://mathbabysteps.blogspot.com/2016/12/affine-communication-lemma.html}{POST}, \href{https://math.stackexchange.com/questions/4146296/declaremathoperator-specspecdoubt-about-nikes-lemma-about-the-intersecti}{POST}.
\end{comment}

\subsection{}

Or we could do this...

Consider the following commutative diagram with corresponding ideals and elementes as labeled.

% https://q.uiver.app/#q=WzAsMyxbMSwwLCJnXFxpblxcbWF0aGZyYWsgcCdcXHRyaWFuZ2xlbGVmdCBCIl0sWzIsMSwiZy8xXFxpblxcbWF0aGZyYWsgcFxcdHJpYW5nbGVsZWZ0IEJfZyJdLFswLDEsImcnXFxpblxcbWF0aGZyYWsgcVxcdHJpYW5nbGVsZWZ0IEFfZiJdLFswLDJdLFswLDFdLFsyLDFdXQ==
\[\begin{tikzcd}
	& {g\in\mathfrak p'\triangleleft B} \\
	{g'\in\mathfrak q\triangleleft A_f} && {g/1\in\mathfrak p\triangleleft B_g}
	\arrow[from=1-2, to=2-1]
	\arrow[from=1-2, to=2-3]
	\arrow[from=2-1, to=2-3]
\end{tikzcd}\]
By commutativity of the diagram, given a prime ideal $\mathfrak p$ then it corresponding uniquely to $\mathfrak p'$ and $\mathfrak q$; given $g/1$, then there exist $g$ and $g'$.
\begin{align*}
	\operatorname{Spec}B_g 
	\cong &~ \{\mathfrak p'\triangleleft B ~\vert~ g\notin \mathfrak p'\}\\
	\cong &~ \{\mathfrak q\triangleleft A_f ~\vert~ g'\notin \mathfrak q\}\\
	\cong &~ \operatorname{Spec}A_f\setminus \{[\mathfrak p] ~:~ g'\in \mathfrak p\}=\operatorname{Spec}(A_f)_{g'}.
\end{align*}

\section{}
\textit{Let $X$ be a scheme with \textbf{affine} open subsets $V\subset U\subset X$. Given $f\in \Gamma (U,\mathcal O_X)$ such that $\mathbf D_U(f)\subset V\subset U\subset X$. Then $\mathbf D_U(f)=\mathbf D_V(f\vert_V)$.}\todo{?}
\begin{comment}
\begin{proof}
It suffices to check $\mathbf V_U(f)=\mathbf V_V(f\vert_V)$. 
\begin{align*}
	\mathbf V_U(f) = &~ \{\mathfrak p\in U ~\vert~ f\in\mathfrak p\}\\
	= &~ \{\mathfrak p\in V ~\vert~ f\vert_V\in \mathfrak p\}\\
	= &~ \mathbf V_V(f\vert_V)
\end{align*}
\end{proof}
\end{comment}

\section{5.3.2. Affine Communication Lemma.}

See this \href{http://mathbabysteps.blogspot.com/2016/12/affine-communication-lemma.html}{POST}.

Question: why stalk-local implies affine-local?

\section{5.3.A.}
\textit{Show that locally Noetherian schemes are quasi-separated.}

\begin{proof}
	According to \Cref{5.1.H.}, it suffices to show the existence of a cover of $X$ by affine open satisfies certain property. 
\end{proof}