\section{8.1.1.}

\subsection{Remarks}
See a post discussing \enquote{Does "local on the target" mean the same thing as "local on the base"?} \href{https://math.stackexchange.com/questions/2767806/does-local-on-the-target-mean-the-same-thing-as-local-on-the-base}{HERE}.

\section{8.1.A.}

\begin{proof}
Notice that $X\times_S Y'\simeq X\times_Y Y\times_S Y'$ and $X'\times_{Y'} X\times_S Y'\simeq X'\times_{Y'}Y'\times_S X\simeq X\times_S X'$. Therefore apply (ii) and (i) we can conclude $X\times_S X'\to Y\times_S Y'$ has property $P$.
\end{proof}


\section{8.1.B.}

\begin{proof}
    Requirement (i) is clear. 
    
    For (ii), we have the following diagram % https://q.uiver.app/#q=WzAsNCxbMCwwLCJYXFx0aW1lc19ZIFknIl0sWzAsMSwiWCJdLFsxLDEsIlkiXSxbMSwwLCJZJyJdLFsxLDIsIlxcc2ltZXEiXSxbMywyXSxbMCwxXSxbMCwzXV0=
\[\begin{tikzcd}
	{X\times_Y Y'} & {Y'} \\
	X & Y
	\arrow[from=1-1, to=1-2]
	\arrow[from=1-1, to=2-1]
	\arrow[from=1-2, to=2-2]
	\arrow["\simeq", from=2-1, to=2-2]
\end{tikzcd}\]
Observe that $X\times_Y Y'\simeq Y\times_Y Y'\simeq Y'$. Hence (ii) is satisfied. 

For (iii), part (a) is clear. It remains to consider part (b). And for this part we refer to \cite{hartshorne2013algebraic} Exercise 2.17. (a) on Page 81 of Chapter 2. 
\end{proof}

\section{8.2.B.}
\subsection{(a)}
See this post \href{https://math.stackexchange.com/questions/3137459/kx-x-is-not-integral-over-kx}{HERE} on StackExchange.
To show a ring morphism is not integral, we can consider the lying over property. 

\section{8.3.B.}

\subsection{(a)}
	See 5.3.4. \cite{RaviRisingSea}.
\begin{proof}
	Notice that any morphism of schemes consists of a continuous map such that 
	\[\pi^{-1}(U)\subset X\]
	is open. Then we apply Exercise 5.1.C., which shows that $X$ is a Noetherian topological space. Apply Exercise 3.6.U. shows that any open subset, including $\pi^{-1}(U)$ is quasi-compact. And this proves that this morphism $\pi$ is quasi-compact.
\end{proof}

\subsection{(b)}

\todo{?}

\section{8.3.D.}

Clear. Because affine scheme is both quasi-compact and quasi-separated.

\section{8.3.F.}

For example (3), see Exercise 9.12. by Gathmann notes on Commutative algebra \cite{GathNotes}, Page 83.

\section{8.3.H.}

Apply 8.2.C. and use the definition \enquote{for all}.