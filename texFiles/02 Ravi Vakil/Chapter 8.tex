\section{8.1.1.}

\subsection{Remarks}
See a post discussing \enquote{Does "local on the target" mean the same thing as "local on the base"?} \href{https://math.stackexchange.com/questions/2767806/does-local-on-the-target-mean-the-same-thing-as-local-on-the-base}{HERE}.

\section{8.1.A.}

\begin{proof}
Notice that $X\times_S Y'\simeq X\times_Y Y\times_S Y'$ and $X'\times_{Y'} X\times_S Y'\simeq X'\times_{Y'}Y'\times_S X\simeq X\times_S X'$. Therefore apply (ii) and (i) we can conclude $X\times_S X'\to Y\times_S Y'$ has property $P$.
\end{proof}


\section{8.1.B.}

\begin{proof}
    Requirement (i) is clear. 
    
    For (ii), we have the following diagram % https://q.uiver.app/#q=WzAsNCxbMCwwLCJYXFx0aW1lc19ZIFknIl0sWzAsMSwiWCJdLFsxLDEsIlkiXSxbMSwwLCJZJyJdLFsxLDIsIlxcc2ltZXEiXSxbMywyXSxbMCwxXSxbMCwzXV0=
\[\begin{tikzcd}
	{X\times_Y Y'} & {Y'} \\
	X & Y
	\arrow[from=1-1, to=1-2]
	\arrow[from=1-1, to=2-1]
	\arrow[from=1-2, to=2-2]
	\arrow["\simeq", from=2-1, to=2-2]
\end{tikzcd}\]
Observe that $X\times_Y Y'\simeq Y\times_Y Y'\simeq Y'$. Hence (ii) is satisfied. 

For (iii), part (a) is clear. It remains to consider part (b). And for this part we refer to \cite{hartshorne2013algebraic} Exercise 2.17. (a) on Page 81 of Chapter 2. 
\end{proof}

\section{8.2.B.}
\subsection{(a)}
See this post \href{https://math.stackexchange.com/questions/3137459/kx-x-is-not-integral-over-kx}{HERE} on StackExchange.
To show a ring morphism is not integral, we can consider the lying over property. 

\section{8.3.B.}

\subsection{(a)}
	See 5.3.4. \cite{RaviRisingSea}.
\begin{proof}
	Notice that any morphism of schemes consists of a continuous map such that 
	\[\pi^{-1}(U)\subset X\]
	is open. Then we apply Exercise 5.1.C., which shows that $X$ is a Noetherian topological space. Apply Exercise 3.6.U. shows that any open subset, including $\pi^{-1}(U)$ is quasi-compact. And this proves that this morphism $\pi$ is quasi-compact.
\end{proof}

\subsection{(b)}

\todo{?}

\section{8.3.D.}

Clear. Because affine scheme is both quasi-compact and quasi-separated.

\section{8.3.F.}

For example (3), see Exercise 9.12. by Gathmann notes on Commutative algebra \cite{GathNotes}, Page 83.
\section{8.3.G.}
\begin{proof}
Since finite morphism is by definition affine, it suffices to show the statement for $X=\Spec A$ an affine scheme. While finite morhism is integral, then we know the inclusion $k\hookrightarrow A$ is finite with $A$ being an integral domain. Apply Exericse 3.2.G, we conclude $A$ is a field. 

\end{proof}
\section{8.3.H.}

Apply 8.2.C. and use the definition \enquote{for all}.

\section{8.4.A.}

\begin{proof}[failed...]
	$\Rightarrow$ is immediate. 
	$\Leftarrow$: in general, we can express constructible subset $A$ as a finite intersection of family members 
\begin{align*}
	A = \bigcap_{\leq \infty} A_i
\end{align*}where each $A_i\in \mathscr A$ is a constructible subset. 
\end{proof}

\begin{proof}[sketch, Proposition 10.13. ~\cite{gortz2020algebraic}]
	So the problem here is to prove two equivalent conditions:\begin{enumerate}
	\item smallest subsets satisfying three conditions;
	\item finite union of locally closed subsets; 
	\item finite disjoint union of locally closed subsets.
	\end{enumerate} We'll use the very fact \enquote{smallest}. Our plan is to check that all subsets of (2) satisfy the three conditions in (1): this is immediately clear for it contains all open subsets and are closed under taking finite unions. It remains to check it's closed under taking complement...
\end{proof}

\section{8.4.B.}

\begin{remark}
	My idea was to find a map $X\to \mathbb A^1$ such that $0\subset X$ is clearly not constructible. But I failed. For hint, see \href{https://math.stackexchange.com/questions/4263567/let-k-be-a-field-the-generic-point-of-operatornamespeckx-does-not-form}{HERE}. Be careful for some details. 
\end{remark}

\begin{proof}
	Suppose $\{\eta\}$ is constructible, i.e. we have $\eta=\bigcup_{\leq\infty} U_i\cap V_i$ where each $U_i\cap V_i$ are disjoint. 
	Then we focus on one pair for $\{\eta\}$ is a singleton: let $0=\{\eta\}= U\cap V$ where $U$ is open and $V$ is closed. 
	While $\eta$ is the generic point, then $\eta\mathbb A^1$. Therefore $\{\eta\}=U$ as a finite subset of $\mathbb A^1$ is open. This contradicts the fact that all closed subset of $\mathbb A^1$ are $\emptyset$, finite subsets, and all of the space $\mathbb A^1$.
\end{proof}