\section{Graded Ring}

Say we wish to define variety in some projective space, i.e. some subsets of $\mathbb P^n$ that could be written as zero locus of some polynomials.
If we don't add any restrictions on polynomials, for example some non-homogeneous polynomials. 
Then the zero locus doesn't make sense because it will depend on the representitive. See Gathmann's Notes Remark 6.5, notes on algebraic geometry, \href{https://agag-gathmann.math.rptu.de/class/alggeom-2021/alggeom-2021-c6.pdf}{HERE}.

\section{Proposition 2.2.}

There are two very technical claims need more details. 

The first one is to prove $\varphi(Y)=Z(T')=Z(\alpha (T)).$
Unwrap the notation precisely according to the definition \begin{align*}
    Z(\alpha (T)) ~=&~ \{~ x\in \mathbb A^n ~\mid~ \alpha(g)(x)=0 ~\forall~ g\in T ~\},\\
    \varphi(Y) ~=&~ \{~ \varphi(y) ~\mid~ y\in Y ~\}.
\end{align*}
Notice that $y=[y_0,...,y_n]\in Y\subset \overline{Y}=Z(T)$, therefore $g(y)=0$ for any $g\in T$.
More precisely, we have 
\[\alpha(g)(\varphi(g))=g(1,y_1/y_0,...,y_n/y_n)=0\] given $g(y)=0$ and $g\in T\subset S^h$, which proves $\varphi(Y)\subset Z(\alpha (T))$.

Conversely, let's start with an element $x=(x_1,...,x_n)\in Z(\alpha (T))$. There's an element $y=[1,x_1,...,x_n]\in Y$ such that $\varphi(y)=x$. Hence we've proved the equality.

And the second one is to check $\varphi^{-1}(W)=Z(\beta(T'))\cap U=Z(\beta(\alpha(T)))\cap U$.


\section{Exercise 2.12.}

For all monomial of degree $d$ in $n+1$ variables $x_0,...,x_n$. 
There are \[\binom{n+d}{n}\] many monomials in total. Bars and balls argument: there are $n$ bars and $d$ balls. In $n+d$ many places, any choice of $n$ bars will corresponds to a monomial, therefore $N$ is the total number of monomials possible. While in we wish to consider them in projective space, we must define $N=\binom{n+d}{n}-1$.

See a solution in lecture notes of Frank-Olaf Schreyer \href{https://www.math.uni-sb.de/ag/schreyer/images/PDFs/teaching/ss21_perugia/AlgGeomSlides19.pdf}{HERE}.

A more detailed solution is given \href{http://therisingsea.org/notes/hartshorne1-2.pdf}{HERE}.

\section{Exercise 2.14.}

There's a prompt on S\'andor's Notes, Lecture 13, Homework 2.79.