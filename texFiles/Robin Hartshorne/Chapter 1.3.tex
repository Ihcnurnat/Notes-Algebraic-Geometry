\section{Lemma 3.1.}

See Sandor's Notes Lecture 4, Lemma 2.6.

Closedness can be checked locally. See a post \href{https://math.stackexchange.com/questions/717382/how-to-prove-that-z-is-closed-subset-iff-x-can-be-covered-by-open-subsets}{HERE}.

\subsection{Closedness local criterion}

This is HW 2.3. of Sandor's notes. 

\textit{Let $X$ be a topological space and $W \subset X$ a subset. Then $W$ is closed if and only if for every $P\in X$ there is an open subset $U \subset X$ such that $P \in U$ and $W \cap U \subset U$ is a closed subset in $U$.}

\begin{proof}
	Assume $W$ is closed, we can simply take $U=X$ for any $P$.

	Conversely, we only need to verify that $X\setminus W$ is open. More precisely, we wish to prove that every point $P\in X\setminus W$ has an open neighborhood that contains in $X\setminus W$. This is ensured by Proposition 2.8. on Page 24 of \cite{lee2010introduction}. 

	Now start with an arbitrary point $q\in X\setminus W$, there exists open subset $U_q$ of $X$ such that \[W\cap U_q\subset U_q\] is closed in $U_q$. Then we can take $U_q\setminus W$ as the open neighborhood of $q$ in $X\setminus W$ as expected. Hence we know $X\setminus W$ is open.
\end{proof} 

\section{Remark 3.1.1.}

See Sandor's Notes Lecture 5, Corollary 2.10.
Notice that Hartshorne defined variety to be irreducible.
See a post explaining why the preimage is dense \href{https://math.stackexchange.com/questions/2860498/regular-functions-are-determined-only-up-to-open-sets#:~:text=An%20important%20consequence%20of%20this,dense%2C%20hence%20equal%20to%20X.}{HERE}.

See Lemma 14. on Page 210 of \cite{bosch2013algebraic}.

\section{Definition: Ring of Regular Function}
\href{https://www2.math.ethz.ch/education/bachelor/lectures/fs2016/math/alg_geom/Solution8.pdf}{HERE} is an explicit description on the ring structure of $\mathcal O_{P,Y}$.

\section{Theorem 3.2.}

See S\'andor's Lecture Notes 09, STEP 03.
See solution of problem 3 \href{https://www2.math.ethz.ch/education/bachelor/lectures/fs2016/math/alg_geom/Solution8.pdf}{HERE}.
See a post \href{https://math.stackexchange.com/questions/1337842/with-regards-to-theorem-3-2-in-hartshorne-are-regular-functions-on-a-variety-si}{HERE}, \href{https://math.stackexchange.com/questions/1909153/some-questions-on-hartshornes-theorem-i-3-2}{HERE}.

\subsection{(c)}
\textit{for each $P$, $\mathcal O_P\cong A(Y)_{\mathfrak m _P}$, and $\operatorname{dim}\mathcal O_P=\operatorname{dim} Y$; }

\begin{proof}
We begin with an injective homomorphism $\alpha:A(Y)\to \mathcal O(Y)$.
And we define a map 
\begin{align*}
    A(Y)_{\mathfrak m_P} &\to \mathcal O_{P,Y}\\
    f/g &\mapsto \langle V,\frac{\alpha(f)}{\alpha(g)}\rangle
\end{align*}where $\frac{\alpha(f)}{\alpha(g)}\in\mathcal O(V)$. 
Now we wish to give an explicit description of $V$.
Since $\alpha(f)\in\mathcal O(Y)$, we know there exists an open subset $P\in V_1\subset Y$ such that 
\[\alpha(f)\mid_{V_1}=\frac{h_1}{h_2}\mid_{V_1}\] where $h_1,h_2\in A$ and $0\notin h_2(V_1)$.
Since $\alpha(g)\in\mathcal O(Y)$, we know there exists an open subset $P\in V_2\subset Y$ such that 
\[\alpha(g)\mid_{V_2}=\frac{h_3}{h_4}\mid_{V_2}\] where $h_3,h_4\in A$ and $0\notin h_4(V_2)$.
Here $g\notin \mathfrak m_P$ by definition of localisation, which gives us 
\[g(P)\neq 0 ~\Rightarrow~ \alpha(g)(P)\neq 0 ~\Rightarrow~ ~\exists~ V_3\subset Y,~ \alpha(g)\mid_{V_3}\neq 0.\] 
Then we take $V=V_1\cap V_2\cap V_3$ will suffice to work. This is because for any point $P\in V$, we have 
\[\frac{\alpha(f)}{\alpha(g)}=\frac{h_1h_4}{h_2h_3}\] for $0\notin h_2h_3(V)$.

% https://q.uiver.app/#q=WzAsNCxbMSwwLCJcXG1hdGhjYWwgTyhZKSJdLFswLDAsIkEoWSkiXSxbMiwwLCJcXG1hdGhjYWwgT197UCwgWX0iXSxbMSwxLCJBKFkpX3tcXG1hdGhmcmFrIG1fUH0iXSxbMSwwLCJcXGFscGhhIl0sWzAsMiwiIiwwLHsic3R5bGUiOnsidGFpbCI6eyJuYW1lIjoiaG9vayIsInNpZGUiOiJ0b3AifX19XSxbMSwzXSxbMywyLCIiLDIseyJzdHlsZSI6eyJib2R5Ijp7Im5hbWUiOiJkYXNoZWQifX19XV0=
\[\begin{tikzcd}
	{A(Y)} & {\mathcal O(Y)} & {\mathcal O_{P, Y}} \\
	& {A(Y)_{\mathfrak m_P}}
	\arrow["\alpha", from=1-1, to=1-2]
	\arrow[hook, from=1-2, to=1-3]
	\arrow[from=1-1, to=2-2]
	\arrow[dashed, from=2-2, to=1-3]
\end{tikzcd}\]
The induced map is given by universal property of localisation, for every elements in $A(Y)\setminus \mathfrak m_{P}$ will be mapped to a unit in $\mathcal O_{P,Y}$. 
And the map given by Lecture Notes 09 of Prof. S\'andor satisfy the universal property (it makes the diagram commute and maps elements outside of $\mathfrak m_P$ to units).

\end{proof}

\section{Proposition 3.3.}

See a post \href{https://math.stackexchange.com/questions/787770/hartshorne-propositon-i-3-3}{HERE}.

\section{Lemma 3.6.}

See a post \href{https://math.stackexchange.com/questions/457328/hartshorne-lemma-i-3-6}{HERE}.

Here are some details for proving $x_i\circ \psi$ being regular implies $\psi$ is a morphism:
Coordinate functions means 
\[\psi(p)=(\psi_1(p),\psi_2(p),...,\psi_n(p))\in Y\subset \mathbb A^n\] for any $p\in X$.
Firstly, we check $\psi$ is continuous. Take any closed subset $Z(f_1,...,f_r)\subset Y$ for some polynomial $f_1,...,f_r\in A=k[x_1,...,x_n]$. We can compute the preimage as \begin{align*}
    \psi^{-1}(Z(f_1,...,f_r)) =& \{~ p\in X ~\mid ~\forall~ p\in X,~ f_i\circ \psi(p)=0 ~\}.
\end{align*}
Notice that for any $p\in X$, \[f_i\circ \psi(p) ~=~ f_i(\psi_i(p),...,\psi_n(p))\] is continuous since $f_i$ is a polynomial and each $\psi_i:=x_i\circ \psi$ is continuous by assumption that they're regular.
Notice that the preimage of $\psi$ is precisely intersection of $\psi_i^{-1}(\{0\})$ where $1\leq i\leq n$. Hence the preimage is closed, and it follows that $\psi$ is continuous as expected. 

Secondly, fix an arbitrary open subset $V\subset Y$ with an arbitrary regular function $g:V\to k$, we wish to prove $g\circ \psi:\psi^{-1}(V)\to k$ is regular. For any $\psi(p)\in V$ with some $p\in X$, there exists a neighborhood $\psi(p)\in U\subset Y$ such that $g$ equals to an expression of quotients of polynomial, i.e. \[g=\frac{g_1}{g_2}\] where $g_1,g_2\in A$. Then for $p\in X$, take the open neighborhood of it as $\psi^{-1}(U)$, we can see 
\[g\circ \psi (p) = \frac{g_1(\psi(p))}{g_2(\psi(p))}.\]

\section{Exercise 3.6.}

See a post \href{https://math.stackexchange.com/questions/2120717/there-are-quasi-affine-varieties-which-are-not-affine#:~:text=I%20see%20this%20sentence%20in,(0%2C0)%7D.}{HERE}.

\section{Exercise 3.17.}
\textit{\textbf{Normal Varieties.} A vareity $Y$ is \textbf{normal at a point} $P\in Y$ if $\mathscr O_p$ is an integrally closed ring. $Y$ is \textbf{normal} if it is normal at every point.}

\subsection{(a)}\textit{Show that every conic in $\mathbb P^2$ is normal.}

\begin{proof}
According to Exercise 1.1.(c), we assume conic $Y$ in $\mathbb P^2_{x,y,z}$ is defined by an irreducible homogeneous polynomial of degree $2$. 

And by Exercise 3.1.(c) we know every conic in $\mathbb P^2$ is isomorphic to $\mathbb P^1$. To check it's normal, we need to show for any $P\in Y=\mathbb P^1$, the local ring $\mathscr O_P$ is an integrally closed ring. 

Notice that 
\[I(\mathbb P^1)=\{f\in k[x,y,z]^h ~\mid~ \forall X\in \mathbb P^1,~ f(X)=0\}=\langle x,y\rangle\trianglelefteq k[x,y,z].\] According to Theorem 3.4, we can compute 
\begin{align*}
	\mathscr O_P =& S(Y)_{(\mathfrak m_P)}\\
	=& (k[x,y,z]/I(\mathbb P^1))_{(\mathfrak m_P)}\\
	=& k[z]_{(\mathfrak m_P)}
\end{align*}
This is degree $0$ part of the localisation $k[z]_{\mathfrak m_P}$, which is UFD given localisation preserves UFD and subring $k[z]_{(\mathfrak m_P)}$ is UFD. While UFD is integrally closed ring, so we have every $\mathscr O_P$ is integrally closed. Hence every conic in $\mathbb P^2$ is normal.\\\\

Or we can notice (?)
\[\mathscr O_{P,\mathbb P^1}\simeq \mathscr O_{P,\mathbb A}=A(\mathbb A^1)_{\mathfrak m_P}=(k[x])_{\mathfrak m_P}.\]

\end{proof}

\subsection{(b)}
\textit{Show that the quadric surfaces $Q_1,Q_2$ in $\mathbb P^3$ given by equations $Q_1: xy-zw$; $Q_2:xy=z^2$ are normal (cf. (II. Ex. 6.4) for the latter.)}

\begin{proof}
	Denote $Q_1=Z(xy-zw)\subset \mathbb P^3_{x,y,z,w}$. We have to compute the localisation of its homogeneous coordinate ring at some point $P\in Q_1$
	\begin{align*}
		\mathscr O_P =& S(Q_1)_{(\mathfrak m_P)}\\
		=& (k[x,y,z,w]/I(Q_1))_{(\mathfrak m_P)}\\
		=& (k[x,y,z,w]/\langle xy-zw\rangle)_{(\mathfrak m_P)}\\
	\end{align*}

\end{proof}

\subsection{(c)}
\textit{Show that the cuspidal cubic $y^2=x^3$ in $\mathbb A^2$ is not normal.}

\subsection{(d)}
\textit{If $Y$ is affine, then $Y$ is normal $\Leftrightarrow$ $A(Y)$ is integrally closed.}

\subsection{(e)}
\textit{Let $Y$ be an affine variety. Show that there is a normal affine variety $\widetilde Y$, and a morphism $\pi: \widetilde Y\to Y$, with the property that whenever $Z$ is a normal variety, and $\varphi:Z\to Y$ is a \textbf{dominant} morphism (i.e., $\varphi(Z)$ is dense in $Y$), then there is a unique morphism $\theta:Z\to \widetilde Y$ such that $\varphi=\pi\circ\theta$. $\widetilde Y$ is called the \textbf{normalization} of $Y$. You will need (3.9A) above.}

\section{Exercise 3.18.}
\textit{\textbf{Projectively Normal Varieties.} A projective variety $Y\subset \mathbb P^n$ is \textbf{projectively normal} (with respect to the given embedding) if its homogeneous coordinate ring $S(Y)$ is integrally closed.}

\subsection{(a)}

\section{Exercise 3.20.}
\textit{Let $Y$ be a variety of dimension $\geq 2$, and let $P\in Y$ be a normal point. Let $f$ be a regular function on $Y-P$.}

\subsection{(a)}