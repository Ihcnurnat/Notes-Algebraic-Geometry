\section{Defintion: Presheaf}

\subsection{Two Pathological Examples}

Here are two examples taken from Tennison's \cite{tennison1975sheaf}.

Let $X$ be any topological space with more thatn one point, i.e. $X=\{0,1\}$ or $X=\{0,1\}\to\mathbb R$.

Define a presheaf $\mathscr P_1$ by 
\begin{align*}
    \begin{cases}
        \mathscr P_1(X)=\mathbb Z \\
        \mathscr P_1(U)=0 ~\text{ for open }~ U\subsetneq X\\
        \text{All restrictions except $\rho_{XX}$ being constant maps.}
    \end{cases}.
\end{align*}Here $0$ denotes the trivial Abelian group.

Pick $x_0\in X$. Define a presheaf $\mathscr P_2$ by 
\begin{align*}
    \begin{cases}
        \mathscr P_2(U)=\mathbb Z ~\text{ for $U$ open in $X$ such that $x_0\in U$}\\
        \mathscr P_2(U)=0 ~\text{ for $U$ open in $X$ such that $x_0\notin U$}\\
        \text{restrictions }~ \rho_{UV}=\begin{cases}
            \operatorname{id}_{\mathbb Z} &~\text{ if $x_0\in V\subset U$}\\
            0 & ~\text{ trivial map if not}
        \end{cases}
    \end{cases}.
\end{align*}Here the second appearance of $0$ denotes the trivial map.

\section{Example 1.0.3.}

See some examples of presheaves that are not sheaves \href{https://en.wikipedia.org/wiki/Sheaf_(mathematics)}{HERE}; a post \href{https://math.stackexchange.com/questions/195363/constant-presheaf-not-necessarily-a-sheaf-proof}{HERE}.

In Wiki's page \href{https://en.wikipedia.org/wiki/Sheaf_(mathematics)}{HERE}, it introduced non-separated presheaf, i.e. presheaf that doesn't satisfy locality axiom for sheaf.

\section{Proposition-Definition 1.2.}

See \href{https://stacks.math.columbia.edu/tag/007X}{\textit{Sheafification}} on The Stacks Project.

See solution of problem 3 \href{https://www2.math.ethz.ch/education/bachelor/lectures/fs2016/math/alg_geom/Solution11.pdf}{HERE}.

Of course, consult Ravi's Notes on Sheafification; 

or see Section 6.5 on Page 232 of \cite{bosch2013algebraic}.

Also, see a REU paper \href{http://www.math.uchicago.edu/%7Emay/VIGRE/VIGRE2011/REUPapers/WengD.pdf}{HERE} by Daping Weng.

A short paper by Tom is \href{https://www.maths.ed.ac.uk/~tl/sheaves.pdf}{HERE}.

\section{Definition: Inverse Image Sheaf}

See \href{https://en.wikipedia.org/wiki/Inverse_image_functor}{Wiki} for motivation of such definition.

See a \href{https://en.wikipedia.org/wiki/Inverse_image_functor}{POST} that gives more details, as well as a counterexample.

From Prof. S\'andor's Email: Let $X$ by disjoint union of two copies of $Y$ with a continuous map $f:X\to Y$. Assume $Y$ is irreducible and let $\mathscr G$ be a constant sheaf on $Y$. We claim that $f^{-1}_{\text{pre}} \mathscr G$ is just a presheaf, but not a sheaf. 

Any open subset $W_1, W_2\in X$ will have intersection in $Y$. Then any section will agree on their intersections. Take two sections from $0\amalg Y$ and $Y\amalg Y$, there won't be a global section such that restriction is either of them.

\section{Exercise 1.3.}

See a post \href{https://math.stackexchange.com/questions/1387214/the-induced-map-on-stalks-is-well-defined}{HERE} for explicit information of induced map on stalks. 

See the solution from a post \href{https://math.stackexchange.com/questions/4450406/surjective-morphism-of-sheaves}{HERE}.

See \href{https://www.math.arizona.edu/~cais/CourseNotes/AlgGeom04/Hartshorne_Solutions.pdf}{HERE} for a partial solution, as well as a counterexample. 
\subsection{(a)}

Now assume $\varphi$ is surjective. Fix an open subset $U\subset X$ and a section $s\in\mathscr G(U)$. Now we can pick any point $p\in U$, consider the stalk at it.


\section{Exercise 1.8.}

See Rotman's \cite{rotman2009introduction}, Lemma 6.68. on Page 378.