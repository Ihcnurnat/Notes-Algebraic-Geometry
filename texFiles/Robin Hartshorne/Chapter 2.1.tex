\section{Example 1.0.3.}

See some examples of presheaves that are not sheaves \href{https://en.wikipedia.org/wiki/Sheaf_(mathematics)}{HERE}; a post \href{https://math.stackexchange.com/questions/195363/constant-presheaf-not-necessarily-a-sheaf-proof}{HERE}.

In Wiki's page \href{https://en.wikipedia.org/wiki/Sheaf_(mathematics)}{HERE}, it introduced non-separated presheaf, i.e. presheaf that doesn't satisfy locality axiom for sheaf.

\section{Proposition-Definition 1.2.}

See \href{https://stacks.math.columbia.edu/tag/007X}{\textit{Sheafification}} on The Stacks Project.

See solution of problem 3 \href{https://www2.math.ethz.ch/education/bachelor/lectures/fs2016/math/alg_geom/Solution11.pdf}{HERE}.

Of course, consult Ravi's Notes on Sheafification; 

or see Section 6.5 on Page 232 of \cite{bosch2013algebraic}.

Also, see a REU paper \href{http://www.math.uchicago.edu/%7Emay/VIGRE/VIGRE2011/REUPapers/WengD.pdf}{HERE} by Daping Weng.

A short paper by Tom is \href{https://www.maths.ed.ac.uk/~tl/sheaves.pdf}{HERE}.

\section{Definition: Inverse Image Sheaf}

See \href{https://en.wikipedia.org/wiki/Inverse_image_functor}{Wiki} for motivation of such definition.

See a \href{https://en.wikipedia.org/wiki/Inverse_image_functor}{POST} that gives more details, as well as a counterexample.