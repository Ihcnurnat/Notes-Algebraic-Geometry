\section{Exercise 5.1.}

\subsection{(a)}

According to this \href{https://www.desmos.com/calculator/08sanf19br}{picture}, (a) is a tacnode. 

%Denote $f_1:=x^2-x^4-y^4$, . Hence it's an integral domain, then we know $\langle f_1\rangle$ is a prime ideal. So we can choose generator $f_1$ for the affine variety
Denote $f_1:=x^2-x^4-y^4$, which is irreducible in UFD $k[x,y]$ hence it's prime. We can then compute the ideal defined by this affine variety 
\[I(Z(\langle f_1\rangle))=\sqrt{\langle f_1\rangle}=\langle f_1\rangle.\]
To locate all singular points, we'll use Theorem 4.6 on the notes. So we have to compute the Jacobian matrix of the above affine variety at some point $P\in \mathbb A^2$ on the affine variety. We choose $f_1$ itself as generators for the ideal of the affine variety and compute the Jacobian matrix
\begin{align*}
    J(P) &= \begin{pmatrix}
        \frac{\partial f_1}{\partial x}(P) & \frac{\partial f_1}{\partial y}(P)
    \end{pmatrix}=
            \begin{pmatrix}
                2x-4x^3 (P) & -4y^3 (P)
            \end{pmatrix}.
\end{align*}

The rank of the matrix is $0$. By Krull's Hauptidealsatz, since $f_1\in k[x,y]$ is neither a zero-divisor nor a unit, we know every minimal prime containing $\langle f_1\rangle$ has height $1$. But $\langle f_1\rangle$ is already a prime ideal, so it has height exactly $1$. We're ready to compute the dimension of the affine variety 
\[\operatorname{dim}(Z(f_1))=\operatorname{dim}A(Z(f_1))=\operatorname{dim}(k[x,y])-\operatorname{height}(\langle f_1\rangle)=2-1=1.\]
Therefore we must have the rank of the matrix as $n-d=2-1=1$, contradiction. It follows that $P$ is a singular point.

\subsection{(b)}

According to this \href{https://www.desmos.com/calculator/c298mxdjtn}{picture}, (b) is a node.

Denote $f_2=xy-x^6-y^6$. Similarly, we choose $f_2$ itself as the generator for the affine variety it defined.
Again, we have to compute the Jacobian matrix of $Z(f_2)$ at $P=(0,0)$. 
\begin{align*}
    J(P) &= \begin{pmatrix}
        \frac{\partial f_2}{\partial x}(P) & \frac{\partial f_2}{\partial y}(P)
            \end{pmatrix}
        = \begin{pmatrix}
            y-6x^5 (P) & x-6y^5 (P)
        \end{pmatrix} 
        = \begin{pmatrix}
            0 & 0
        \end{pmatrix},
\end{align*}which as rank $0$. We can compute the dimension of the affine variety as $1$, it follows that $P$ is a singular point given the rank of matrix isn't $2-1=1$.

\subsection{(c)}

We just need to check the Jacobian matrix 
\begin{align*}
    J(P) &= \begin{pmatrix}
        \frac{\partial f_2}{\partial x}(P) & \frac{\partial f_2}{\partial y}(P)
            \end{pmatrix}
        = 
\end{align*}

\subsection{}

See a post \href{https://math.stackexchange.com/questions/1728013/is-the-affine-curve-y2-x4y4-in-mathbb-a2-singular}{HERE}.

See REB's solution \href{https://math.berkeley.edu/~reb/courses/256A/1.5.pdf}{HERE}.

See a post on irreducibility of polynomial over $\mathbb C$ \href{https://math.stackexchange.com/questions/1335827/showing-a-polynomial-is-irreducible-over-mathbbcx-y}{HERE}.