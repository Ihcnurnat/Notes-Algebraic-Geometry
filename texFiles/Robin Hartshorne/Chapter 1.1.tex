
\section{Definition: Irreducible}

For equivalent definitions, see Wiki \href{https://en.wikipedia.org/wiki/Irreducible_component}{HERE};
Also see "dense" on Wiki \href{https://en.wikipedia.org/wiki/Dense_set}{HERE}.

See Atiyah's \cite{atiyah1994introduction} Exercise 19 from Chapter 1 for more information...

See Lemma 14. on Page 210 of \cite{bosch2013algebraic} for equivalent characterisation of \textit{irreducible}.

\section{Example 1.1.3.}
See Atiyah's \cite{atiyah1994introduction} Exercise 19 from Chapter 1, which proves it must be dense.

For the sake of contradiction, assume a non-empty subset $A\subset X$ is reducible. Hence there exist two proper closed subset $A_1,A_2\subset A$ such that $A=A_1\cup A_2$.
Then we have $$X=(A^c\cup A_1)\cup (A^c\cup A_2),$$ which implies $X$ is reducible. \todo{\ding{53}!}
\subsection{}
See a post \href{https://math.stackexchange.com/questions/460074/an-open-subset-of-an-irreducible-set-is-dense}{HERE} \todo{\ding{52}}

First approach applied some density argument. While the second approach, similarly, gave a decomposition of the whole space given the open set is reducible.

\section{Example 1.1.4.}

See Atiyah's \cite{atiyah1994introduction} Exercise 20 from Chapter 1.

\section{Definition.} 
"Induced topology". Definition of \textit{quasi affine variety}, see \href{https://math.stackexchange.com/questions/1086839/a-confusion-regarding-the-definition-of-a-quasi-affine-variety}{HERE}.

\section{Prop 1.2 (d)}

According to Hilbert's Nullstellensatz, I agree we'll get $I(Z(\mathfrak a))\subset \sqrt{\mathfrak a}$.
For the reverse inclusion, pick any $f\in A$ such that $f^r\in\mathfrak a$ where $r\in\mathbb Z_{>0}$. We wish to show that $f(P)=0$ for any $P\in Z(\mathfrak a)$. By definition, $f^r(P)=0$ given $f^r\in\mathfrak a$. And this implies $$f^r(P)=(f(P))^r=0 ~\Rightarrow~ f(P)=0$$ given the polynomial ring $A$ is an integral domain. Therefore we get the inclusion 
$$I(Z(\mathfrak a))\supset \sqrt{\mathfrak a}.$$

See Theorem 6 on Page 183, Strong Nullstellensatz, \cite{cox2013ideals}. \todo{\ding{52}}

\section{Theorem 1.3A. Hilbert's Nullstellensatz}

For the case where $k$ isn't algebraically closed, see \cite{reid1995undergraduate} Remarks in 5.6.

\subsection{}

For the proof, see \cite{kemper2011course} Chapter 1 for details.

The followings are some comments for \cite{kemper2011course} Chapter 1 Theorem 1.7.

A post on surjective preimage for maximal ideal \href{https://math.stackexchange.com/questions/1198414/pull-back-image-of-maximal-ideal-under-surjective-ring-homomorphism-is-maximal}{HERE}.

A post on preimage for maximal ideal (not necessarily surj) \href{https://math.stackexchange.com/questions/882524/in-an-extension-of-finitely-generated-k-algebras-the-contraction-of-a-maximal}{HERE}.

For completeness, a post on preimage of prime ideals \href{https://math.stackexchange.com/questions/409999/prove-that-the-preimage-of-a-prime-ideal-is-also-prime}{HERE}.

A post on image of prime ideals \href{https://math.stackexchange.com/questions/1805457/how-to-prove-that-the-image-of-a-prime-ideal-is-also-a-prime-ideal}{HERE}, \href{https://math.stackexchange.com/questions/292829/a-proof-that-shows-surjective-homomorphic-image-of-prime-ideal-is-prime}{HERE}, and \href{https://math.stackexchange.com/questions/1144695/assume-p-is-a-prime-ideal-s-t-k-subset-p-show-fp-is-a-prime-ideal}{HERE}.

See Kemper's \cite{kemper2011course}, Lemma 1.22 on Page 17, which completely described prime and maximal ideals in quotients.

\section{Definition. Height}

Here the definition \textit{height} is specifically for prime ideal $\mathfrak p \vartriangleleft_{pr} R$ for some ring $R$. For a general definition, see a post \href{https://math.stackexchange.com/questions/2689141/definition-of-height-of-an-ideal}{HERE}; see a webpage \href{https://encyclopediaofmath.org/wiki/Height_of_an_ideal}{HERE}; or see \cite{kemper2011course} Definition 6.10 on Page 68.

\section{Proposition 1.7.}

Difference between algebraic set and affine algebraic set? See \href{https://math.stackexchange.com/questions/3486407/difference-between-algebraic-set-and-affine-algebraic-set-in-hartshorne}{HERE}.

For an analogue in Projective, see Exercise 2.6 \ref{} in Chapter 1.2. And a post \href{https://math.stackexchange.com/questions/670156/hartshorne-exercise-2-6}{HERE}.

\section{Theorem 1.8A.}

For transcendence degree, see \href{https://en.wikipedia.org/wiki/Transcendental_extension#CITEREFMilne}{HERE} and a \href{https://www.jmilne.org/math/CourseNotes/FT.pdf}{NOTE} by Milne James.

\section{Proposition 1.10.}

Apart from the proof, we discussed \textit{locally closed subset}.
See \href{https://en.wikipedia.org/wiki/Locally_closed_subset}{HERE} for its equivalent definitions.

\section{Proposition 1.13.}

See \href{https://math.stackexchange.com/questions/1292811/help-with-proposition-1-13-in-hartshornes-algebraic-geometry}{HERE}.

\section{Exercise 1.1.}

\subsection{(a)}

By definition of affine coordinate ring we have 
\[A(Y)=A/I(Y)=A/I(Z(f))=A/\sqrt{\langle f\rangle}.\]
While $f$ is irreducible in U.F.D. $k[x,y]$, the ideal it generated will be a prime ideal, which is radical. Therefore we can further simplify the expression as 
\[A/\sqrt{\langle f\rangle}=A/\langle f\rangle=k[x,y]/\langle y-x^2\rangle=k[x].\]
Hence we can conclude $A(Y)$ is isomorphic to a polynomial ring in one variable over $k$.

\subsection{(b)}
