\section{Definition: Dominant Rational Map}

\subsection{}
Well-definess for a rational map being \textit{dominant}.

One thing important to keep in mind is both varieties $X,Y$ are a priori irreducible. There's a completely point-set topological argument \href{https://math.stackexchange.com/questions/182037/dominant-rational-maps}{HERE}. Here a technical detail is "The image of a dense subset under a surjective continuous function is again dense", which is from Wiki's entry. For details of this technicality, see \href{https://math.stackexchange.com/questions/3452534/image-of-a-dense-set-via-a-continuous-surjective-function-is-dense}{HERE}.

See a post on equivalent definition for dominant rational map \href{https://math.stackexchange.com/questions/2557825/equivalent-definitions-of-dominant-rational-map}{HERE}.

A good lecture note \href{https://math.mit.edu/~mckernan/Teaching/09-10/Autumn/18.725/l_14.pdf}{HERE}.

Wiki's entry for \href{https://en.wikipedia.org/wiki/Rational_mapping}{Rational Map}.

Very good note by Vakil \href{https://math.stanford.edu/~vakil/725/class13.pdf}{HERE}.

And a post \href{https://math.stackexchange.com/questions/1843113/how-is-a-dominant-rational-map-well-defined}{HERE}. However, it appears the way used in the proof implicitly required morphism is an open map? So I doubt...

\subsection{}
Some posts share a technical details from general topology. The following statement is taken from Wiki... 

\textit{The image of a dense subset under a surjective continuous function is again dense. More precisely, assume $f:X\to Y$ with $E$ dense in $X$, then $f(E)$ is dense in $f(X)$.}

\begin{proof}
    By definition we have $E\subset f^{-1}(f(E))\subset f^{-1}(\overline{f(E)})$, which is closed given $f$ is continuous. 
    It follows that \[\overline{E}\subset f^{-1}(\overline{f(E)}) ~\Rightarrow~ f(\overline{E})\subset \overline{f(E)}.\]
    Conversely,
    \[f(X)=\overline{f(E)}\cap f(X)\supset f(X)\cap f(X)=f(X) ~\Rightarrow~ f(X)\supset \overline{f(E)}\cap f(X).\]
    Here $\overline{f(E)}$ denotes closure of $f(E)$ in $Y$, while it's intersection with $f(X)$ is the whole $f(X)$, then $f(E)$ is dense in $f(X)$. 
\end{proof}

\subsection{A Pathological Example}

Another equivalent statement required "surjectivity" and say $f(E)$ is dense in $Y$. It's curtial. Also we can only conclude $f(E)$ is dense merely in $f(X)$ instead of $Y$. Since we have the continuous inclusion map $\iota:\mathbb R\to\mathbb C$, then $\operatorname{id}(\mathbb Q)=\mathbb Q$ is just dense in $\mathbb R$ but not dense in $\mathbb C$.

\subsection{}

Say we start with a dominant rational map $\varphi:X\to Y$ with two representatives 
\[\langle U,f\rangle,~\langle V,g\rangle.\]
By definintion of dominant, we know $f(U)$ is dense in $Y$. To check this definition is indepedent of the choice of the representative, we have to check $g(V)$ is dense in $Y$.

Notice that 
\begin{align*}
    Y=\overline{f(U)}=\overline{f(\overline{U\cap V}\cap U)}\subset\overline{f(\overline{U\cap V})}\subset \overline{\overline{f(U\cap V)}}=\overline{g(U\cap V)}\subset \overline{g(V)}.
\end{align*} for $X$ is irreducible and both $U,V$ are non-empty and open then $X=\overline{U\cap V}$. Here the third inclusion is given by the previous technical lemma.

\subsection{Composing Dominant Rational Maps}

See a post \href{https://math.stackexchange.com/questions/459827/how-to-define-the-composition-of-two-dominant-rational-maps}{HERE}.

% https://q.uiver.app/#q=WzAsNSxbMCwwLCJYIl0sWzEsMCwiWSJdLFsyLDAsIloiXSxbMCwxLCIoVSxcXHBoaV9VKSJdLFsxLDEsIihWLFxccHNpX1YpIl0sWzAsMSwiXFxwaGkiLDAseyJzdHlsZSI6eyJib2R5Ijp7Im5hbWUiOiJkYXNoZWQifX19XSxbMSwyLCJcXHBzaSIsMCx7InN0eWxlIjp7ImJvZHkiOnsibmFtZSI6ImRhc2hlZCJ9fX1dXQ==
\[\begin{tikzcd}
	X & Y & Z \\
	{(U,\phi_U)} & {(V,\psi_V)}
	\arrow["\phi", dashed, from=1-1, to=1-2]
	\arrow["\psi", dashed, from=1-2, to=1-3]
\end{tikzcd}\]

To prove the composition is a dominant rational map, we need to find a representative. 
We define 
\[W:=\phi_U^{-1}(V).\] \todo{? slightly different than the post online}
And we claim $(W,~\psi_V\circ\phi_U)$ will be suitable for a representative for $\psi\circ\phi$.
First of all, notice that $W$ is non-empty. This is because $\phi_U(U)\cap V\neq \emptyset$ given $\phi_U(U)$ is dense in $Y$ and $V$ is assumed to be non-empty open subset. While $Y$ is irreducible, by Lemma 14. of \ref{bosch2013algebraic} on Page 210, which states that $\phi_U(U)\cap V$ is nontrivial. By definition this implies 
\[\phi_U^{-1}(V)\neq \emptyset.\] This is non-empty and open. Note $X$ is irreducible, hence $W=\phi_U^{-1}(V)$ is dense in $X$. Hence $\psi_V\circ\phi_U (W)$ is dense in $Z$ given both maps are continuous by being a morphism.

\subsection{}

See a post \href{https://math.stackexchange.com/questions/1191794/composition-of-dominant-rational-maps}{HERE}, \href{https://people.maths.bris.ac.uk/~malab/PDFs/Algebraic%20Geometry%20L3.pdf}{HERE}, and \href{https://math.stackexchange.com/questions/431578/questions-about-the-composition-of-two-dominant-rational-maps}{HERE}.


\section{Lemma 4.2.}

See a post \href{https://math.stackexchange.com/questions/4552630/hartshorne-algebraic-geometry-proof-of-lemma-4-2-chapter-i-section-4-on-ratio}{HERE}.

\section{Definition Blowing Up}

On Page 29, Hartshorne defined blowing-up of a closed affine subvariety $Y$ of $\mathbb A^n$ passing through $O$.

See S\'andor's Notes, Lecture 22, \textit{strict transform}.

See Daniel's notes \href{http://therisingsea.org/notes/hartshorne1-4.pdf}{HERE}. Here the notation $(\cdot)^{-}$ stands for taking closure.

\section{Exercise 4.1.}\label{Chapter 1, Ex. 4.1.}

Define a function 
\begin{align*}
    h:U\cup V &\to k ~\text{ by }~\\
     x &\mapsto \begin{cases} f(x) & ~\text{ when }~ x\in U\setminus V\\
                                g(x) & ~\text{ when }~ x\in V
                \end{cases}.
\end{align*} Since $f=g$ on $U\cap V$, hence the function $h$ is well-defined. 
For any point $p\in U\cup V$, if $p\in U$, then we apply assumption that $f$ is regular. For $p\in V$, similarly apply assumption that $g$ is regular.
Hence $h$ is regular on $U\cup V$.

Let $f$ be a rational function on $X$. So we take all equivalence class $\{\langle U_i,f_i\rangle\}_{i\in I}$ that represents $f$.
By the above lemma and the definition of regular function, there's a regular function $g$ that's defined on $U:=\bigcup_{i\in I}U_i$ that extends all $f_i$. 
Therefore we can take a representative of $f$ as 
\[ \langle U,g\rangle.\] Note $\langle U,g\rangle=\langle U_i,f_i\rangle$ by definition, hence it's indeed a representative of $f$. 

Also $U$ is the largest open set. Suppose it's not, then we have $\langle U_{i_0},f_{i_0}\rangle$ represents $f$ such that $U_{i_0}\supsetneq U$. And this will contradicts the construction of $U$, which must contain $U_{i_0}$.

\section{Exercise 4.2.}

We're given a rational map $\varphi:X\dashrightarrow Y$. Suppose we have two equivalent representatives
\[ \langle U,\varphi_U\rangle,~\langle V,\varphi_V\rangle.\] This means two morphisms $\varphi_U,\varphi_V$ agree on $U\cap V$.

It suffices to prove that we can define a morphism $\psi:U\cup V\to Y$ that extends both $\varphi_U$ and $\varphi_V$. 
Similarly, we can apply argument of \ref{Chapter 1, Ex. 4.1.} to conclude the existence of a largest open set on which $\varphi$ is represented by a morphism.

Both $\varphi_U,\varphi_V$ are continuous function that agree on their intersection, then we can define 
\begin{align*}
    \psi:U\cup V &\to Y ~\text{ by }~ \psi(x)=\begin{cases}
        \varphi_U(x) & ~\text{ when }~ x\in U\setminus V\\
        \varphi_V(x) & ~\text{ when }~ x\in V 
    \end{cases}.
\end{align*}
Clearly it's a well-defined continuous function. For any open subset $W\subset Y$ with an arbitrary regular function $f:W\to k$. 
We have $f\circ \psi:\psi^{-1}(W)\to k$ is a regular function since it extends both 
\[f\circ \varphi_U:\varphi_U^{-1}(W)\to k,~ f\circ \varphi_V:\varphi_V^{-1}(W)\to k.\]
While both two regular functions agree on their intersection, then we can conclude using \ref{Chapter 1, Ex. 4.1.} that $f\circ\psi$ is a regular function. 
And this proves that $\psi$ is indeed a morphism on $U\cup V\to Y$.