\section{Defintion: Presheaf}

\subsection{Two Pathological Examples}

Here are two examples taken from Tennison's \cite{tennison1975sheaf}.

Let $X$ be any topological space with more thatn one point, i.e. $X=\{0,1\}$ or $X=\{0,1\}\to\mathbb R$.

Define a presheaf $\mathscr P_1$ by 
\begin{align*}
    \begin{cases}
        \mathscr P_1(X)=\mathbb Z \\
        \mathscr P_1(U)=0 ~\text{ for open }~ U\subsetneq X\\
        \text{All restrictions except $\rho_{XX}$ being constant maps.}
    \end{cases}.
\end{align*}Here $0$ denotes the trivial Abelian group.

Pick $x_0\in X$. Define a presheaf $\mathscr P_2$ by 
\begin{align*}
    \begin{cases}
        \mathscr P_2(U)=\mathbb Z ~\text{ for $U$ open in $X$ such that $x_0\in U$}\\
        \mathscr P_2(U)=0 ~\text{ for $U$ open in $X$ such that $x_0\notin U$}\\
        \text{restrictions }~ \rho_{UV}=\begin{cases}
            \operatorname{id}_{\mathbb Z} &~\text{ if $x_0\in V\subset U$}\\
            0 & ~\text{ trivial map if not}
        \end{cases}
    \end{cases}.
\end{align*}Here the second appearance of $0$ denotes the trivial map.

\section{Example 1.0.3.}

See some examples of presheaves that are not sheaves \href{https://en.wikipedia.org/wiki/Sheaf_(mathematics)}{HERE}; a post \href{https://math.stackexchange.com/questions/195363/constant-presheaf-not-necessarily-a-sheaf-proof}{HERE}.

In Wiki's page \href{https://en.wikipedia.org/wiki/Sheaf_(mathematics)}{HERE}, it introduced non-separated presheaf, i.e. presheaf that doesn't satisfy locality axiom for sheaf.
\section{Proposition 1.1.}

Given $s_P=0\in\mathscr F_P$, we know $0\in \mathscr F_P$ could be represented by $\langle W_1,0\rangle$ and $s_P=\langle U,s\rangle$ for two open subsets $P\in W_1,U$.They're equal means there exist some open $W_P\subset U\cap W_1$ such that $s\vert_{W_P}=0\vert_{W_P}=0$. While $P$ is arbitrary, therefore we have an open covering of $U$ by $\{W_P\}_{P\in U}$. Then we can apply sheaf property to conclude $s=0$ as desired.

\section{Proposition-Definition 1.2. Sheafification}

See \href{https://stacks.math.columbia.edu/tag/007X}{\textit{Sheafification}} on The Stacks Project.

See solution of problem 3 \href{https://www2.math.ethz.ch/education/bachelor/lectures/fs2016/math/alg_geom/Solution11.pdf}{HERE}.

Of course, consult Ravi's Notes on Sheafification; 

Or see Section 6.5 on Page 232 of \cite{bosch2013algebraic}.

Also, see a REU paper \href{http://www.math.uchicago.edu/%7Emay/VIGRE/VIGRE2011/REUPapers/WengD.pdf}{HERE} by Daping Weng.

A short paper by Tom is \href{https://www.maths.ed.ac.uk/~tl/sheaves.pdf}{HERE}.

A \href{https://math.stackexchange.com/questions/889497/sheafification-definition}{post} discussing two equivalent definitions of sheafification: one based on compatible germs, and the other one is based on \text{continuous sections}.

\subsection{Isomorphism on stalk}

This is for Lemma 007Z of \href{https://stacks.math.columbia.edu/tag/007X}{Stacks Project}. Similar contents could be found in Rising Sea 2.4.L.

% https://q.uiver.app/#q=WzAsNSxbMSwwLCJzXFxpblxcbWF0aHNjciBGKFUpIl0sWzIsMCwiXFxtYXRoc2NyIEZee1xcc2hhcnB9KFUpXFxuaSAoc191KV97dVxcaW4gVX0iXSxbMSwxLCIoVSx4KVxcaW4gXFxtYXRoc2NyIEZfeCJdLFsyLDEsIlxcbWF0aHNjciBGX3hee1xcc2hhcnB9XFxuaSAoVSwoc191KV97dVxcaW4gVX0pIl0sWzAsMSwidFxcaW5cXG1hdGhzY3IgRihXKSJdLFswLDFdLFswLDJdLFsxLDNdLFsyLDMsIlxcZXhpc3RzICEiLDAseyJzdHlsZSI6eyJib2R5Ijp7Im5hbWUiOiJkYXNoZWQifX19XSxbMCw0XV0=
\[\begin{tikzcd}
	& {s\in\mathscr F(U)} & {\mathscr F^{\sharp}(U)\ni (s_u)_{u\in U}} \\
	{t\in\mathscr F(W)} & {(U,x)\in \mathscr F_x} & {\mathscr F_x^{\sharp}\ni (U,(s_u)_{u\in U})}
	\arrow[from=1-2, to=1-3]
	\arrow[from=1-2, to=2-2]
	\arrow[from=1-3, to=2-3]
	\arrow["{\exists !}", dashed, from=2-2, to=2-3]
	\arrow[from=1-2, to=2-1]
\end{tikzcd}\]

According to Stacks Project, injectivity is proved. 

We focus on proving the surjectivity of $\mathscr F_x\to \mathscr F^{\sharp}_x$. We constructed the induced map on stalks of presheaves based on universal property of $\mathscr F_x$ being a colimit. Therefore the map is unique. And we wish to get an explicit description of that. While for the diagram we know all but the induced map explicity, therefore the map defined such that making the diagram commute must be the induced map by uniqueness of universal property. It follows that we have 
\begin{align*}
    \mathscr F_x &\to \mathscr F^{\sharp}_x \\
    (U,s) &\mapsto (U, (s_u)_{u\in U})
\end{align*} where $x\in U\subset X$, $s\in \mathscr F(U)$, and $(s_u)_{u\in U}\in \prod_{x\in U} \mathscr F_x$ is a compatible germ. 

We could also check this map is well-defined. But it must be for it makes the diagram commute, hence it's the unique induced map on stalk for a given $x\in X$.\todo{do we care?}

The facts that $x\in U$ and $(s_u)_{u\in U}$ is a compatible germ enable us, by definition of sheafification, find: \begin{itemize}
    \item a neighborhood $W\subset U$ such that $x\in W$;
    \item a section $t\in\mathscr F(W)$ such that 
    $t_w=s_w$ for any $w\in W$.
\end{itemize}

And we claim $(W,t)\in \mathscr F_x$ will be the preimage of $(U,(s_u)_{u\in U})$. Now we're ready to compute the image of $(W,t)$ as 

% https://q.uiver.app/#q=WzAsNCxbMCwxLCIoVyx0KVxcaW5cXG1hdGhzY3IgRl94Il0sWzAsMCwidFxcaW5cXG1hdGhzY3IgRihXKSJdLFsxLDAsIih0X3UpX3t1XFxpbiBVfVxcaW4gXFxtYXRoc2NyIEZee1xcc2hhcnB9KFcpIl0sWzEsMSwiKFUsKHRfdSlfe3VcXGluIFV9KVxcaW4gXFxtYXRoc2NyIEZee1xcc2hhcnB9X3giXSxbMSwyXSxbMSwwXSxbMiwzXSxbMCwzLCIiLDIseyJzdHlsZSI6eyJib2R5Ijp7Im5hbWUiOiJkYXNoZWQifX19XV0=

% https://q.uiver.app/#q=WzAsNCxbMCwxLCIoVyx0KVxcaW5cXG1hdGhzY3IgRl94Il0sWzAsMCwidFxcaW5cXG1hdGhzY3IgRihXKSJdLFsxLDAsIih0X3UpX3t1XFxpbiBVfVxcaW4gXFxtYXRoc2NyIEZee1xcc2hhcnB9KFcpIl0sWzEsMSwiKFcsKHRfdSlfe3VcXGluIFd9KVxcaW4gXFxtYXRoc2NyIEZee1xcc2hhcnB9X3giXSxbMSwyXSxbMSwwXSxbMiwzXSxbMCwzLCIiLDIseyJzdHlsZSI6eyJib2R5Ijp7Im5hbWUiOiJkYXNoZWQifX19XV0=
\[\begin{tikzcd}
	{t\in\mathscr F(W)} & {(t_u)_{u\in U}\in \mathscr F^{\sharp}(W)} \\
	{(W,t)\in\mathscr F_x} & {(W,(t_u)_{u\in W})\in \mathscr F^{\sharp}_x}
	\arrow[from=1-1, to=1-2]
	\arrow[from=1-1, to=2-1]
	\arrow[from=1-2, to=2-2]
	\arrow[dashed, from=2-1, to=2-2]
\end{tikzcd}\]

We claim that \[(W,(t_u)_{u\in W})=(U,(s_u)_{u\in U})\in \mathscr F^{\sharp}_x\]
This is because there exists $W\subset W\cap U$ such that \begin{align*}
    (s_u)_{u\in U}\mid_W =& (s_u)_{u\in W}\\
    =& (t_u)_{u\in W}\\
    =& (t_u)_{u\in W}\mid_W.
\end{align*}Here the first and third equality is given by restriction map. The second equality holds: by definition we claimed that $s_w=t_w$ for any $w\in W$. Therefore we've checked that $(W,t)$ is the preimage for an arbitrary element $(U,(s_u)_{u\in U})\in \mathscr F^{\sharp}_x$, and it follows that $\mathscr F\to\mathscr F^{\sharp}$ is surjective.

\subsection{Hint}

The above approach is based on Stacks Project. But Proposition 2.24. on Page 53 of Algebraic Geometry I: Schemes \cite{gortz2020algebraic} gives another solution, which is \textbf{much} more efficient!

It identitifies, in the sense of colimi, that
\[\operatorname{colim}\mathscr F^{\sharp}=\mathscr F_x.\]
  
\section{Definition: Image Sheaf}

See a \href{https://math.stackexchange.com/questions/3838804/image-presheaf-which-is-not-a-sheaf}{post} explaining an example of image presheaf that's not a sheaf.

See a very good \href{https://www2.math.ethz.ch/education/bachelor/lectures/fs2016/math/alg_geom/Solution12.pdf}{solution} for details.

See another example on Page 14 of a brilliant notes \href{https://pi.math.cornell.edu/~apatotski/7400-notes.pdf}{HERE} by Yuri.

\section{Definition: Inverse Image Sheaf}

See \href{https://en.wikipedia.org/wiki/Inverse_image_functor}{Wiki} for motivation of such definition.

See a \href{https://en.wikipedia.org/wiki/Inverse_image_functor}{POST} that gives more details, as well as a counterexample.

From Prof. S\'andor's Email: Let $X$ by disjoint union of two copies of $Y$ with a continuous map $f:X\to Y$. Assume $Y$ is irreducible and let $\mathscr G$ be a constant sheaf on $Y$. We claim that $f^{-1}_{\text{pre}} \mathscr G$ is just a presheaf, but not a sheaf. 

Any open subset $W_1, W_2\in X$ will have intersection in $Y$. Then any section will agree on their intersections. Take two sections from $0\amalg Y$ and $Y\amalg Y$, there won't be a global section such that restriction is either of them.

\section{Definition: Restriction}

See Rising Sea \cite{RaviRisingSea} Exercise 2.7.D. \Cref{2.7.D.} and Example 2.2.8. The definition is very general, it's for both open and closed subsets. Usually, the definition given in Example 2.2.8. of \cite{RaviRisingSea} is easily to work with. And Exercise 2.7.D, states that we can avoid taking inverse image sheaf when the subset is \textit{open}.

\section{Exercise 1.1.}

See "Rising Sea" by Ravi Exercise 2.2.E. on Page 74. It gave another potentially equivalent definition of constant sheaf, which could be easier to check it is indeed sheafification of constant pre-sheaf.

Let $\mathscr F$ denotes the constant pre-sheaf. Then we can compute, by using universal property of colimit that stalk at $p\in U$ is $\mathscr F_p=S$. 

Now we have a concrete description of the compatible germs as 
\begin{align*}
    \mathscr F^{\sharp}(U)=&~\{s:U\to\amalg_{p\in U}\mathscr F_p ~\mid~ ...\}\\
    =&~\{s:U\to \amalg_{p\in U}S ~\mid~ \\
    &\forall~ p\in U,~ \exists~ \text{an open neighorbood}~ V\subset U~ \text{containing}~ p \text{ and } \exists~ t\in \mathscr F(V).\\
    &\text{such that}~ s(q)=s_q=t_q ~\forall~ q\in V.\}
\end{align*}
But notice that for constant presheaf $\mathscr F$, all restriction maps and natural map to stalk is identity as
% https://q.uiver.app/#q=WzAsMyxbMiwwLCJcXG1hdGhzY3IgRigpPVMiXSxbMCwwLCJcXG1hdGhzY3IgRigpPVMiXSxbMSwxLCJcXG1hdGhzY3IgRl9QPVMiXSxbMSwwLCJcXHRleHR7aWR9Il0sWzEsMiwiXFx0ZXh0e2lkfSIsMl0sWzAsMiwiXFx0ZXh0e2lkfSJdXQ==
\[\begin{tikzcd}
	{\mathscr F()=S} && {\mathscr F()=S} \\
	& {\mathscr F_P=S}
	\arrow["{\text{id}}", from=1-1, to=1-3]
	\arrow["{\text{id}}"', from=1-1, to=2-2]
	\arrow["{\text{id}}", from=1-3, to=2-2]
\end{tikzcd}\]
So we can simplify the expression as 
\begin{align*}
    \mathscr F^{\sharp}(U)=&~\{f:U\to \amalg_{p\in U}S ~\mid~ \\
    &\forall~ p\in U,~ \exists~ \text{an open neighorbood}~ V\subset U~ \text{containing}~ p \text{ and } \exists~ t\in S.\\
    &\text{such that}~ f(q)=t_q=\operatorname{id}(t)=t ~\forall~ q\in V\}.
\end{align*}
We claim there's a bijection between two sets 
\[\Phi:\mathscr F^{\sharp}(U)\to\underline{S}(U)\]
For a given $f\in\mathscr F^{\sharp}(U)$, we can define a map $g:U\to S$ by 
\[g(p):= \operatorname{pr}_{p}\circ f(p)\] where the projection map is $\operatorname{pr}_{p}:\amalg_{i\in I} S_i\to S_p$ defined by projection to $p$-coordinate. For any $p\in U$, by definition of sheafification we know there is an open neighborhood $p\in V\subset U$ and $t\in S$ such that 
\[g(q)=\operatorname{pr}_{q}\circ f(q)=t\] for any $q\in V$. And this is precisely saying $g$ is locally constant. Different choices of $f$ will result in different $g$, therefore it's injective. 

Clearly, if we're given a locally constant map $g:U\to S$, we can form a tuple indexed by $p\in U$ as
\[\prod_{u\in U} (g(u))\in \amalg_{u\in U}S.\] And this corresponds to a function in $\mathscr F^{\sharp}(U)$ that satisfies the requirements precisely because $g$ is locally constant. Therefore it's surjective.

What we've shown is that there's a bijection between 
\[\Phi:\mathscr F^{\sharp}(U)\to\underline{S}(U)\] where $\mathscr F$ is constant pre-sheaf. Therefore we conclude that constant sheaf is indeed the sheafification of constant pre-sheaf.

\subsection{References}

See a post \href{https://math.stackexchange.com/questions/249961/sheafification-of-the-constant-presheaf}{HERE}, \href{https://math.stackexchange.com/questions/3834390/sheafification-of-constant-presheaf}{HERE}.

Basically, we need to prove $\underline{S}_{\text{pre}}^{\sharp}\simeq \underline{S}$. And I did by exhibiting a bijection on when they both evaluate at an open subset $U$, i.e. I checked isomorphism between two functors by showing the natural transformation is natural isomorphism.
I didn't do this on stalk for I'm afraid it could be more complicated.\todo{But could we?}

\section{Exercise 1.2.}

\subsection{(a)}

See 2.6.A. Exercise \Cref{2.6.A.} of \cite{RaviRisingSea}.

\subsection{Hint}

A categorical approach is to apply Lemma 002W of \href{https://stacks.math.columbia.edu/tag/04AX}{Tag 04AX}.

\section{Exercise 1.3.}

See a post \href{https://math.stackexchange.com/questions/1387214/the-induced-map-on-stalks-is-well-defined}{HERE} for explicit information of induced map on stalks. 

See the solution from a post \href{https://math.stackexchange.com/questions/4450406/surjective-morphism-of-sheaves}{HERE}.

See \href{https://www.math.arizona.edu/~cais/CourseNotes/AlgGeom04/Hartshorne_Solutions.pdf}{HERE} for a partial solution, as well as a counterexample. 
\subsection{(a)}

Now assume $\varphi$ is surjective. Fix an open subset $U\subset X$ and a section $s\in\mathscr G(U)$. Now we can pick any point $p\in U$, consider the stalk at it.


\section{Exercise 1.8.}

See Rotman's \cite{rotman2009introduction}, Lemma 6.68. on Page 378.

\section{Exercise 1.9. Direct Sum}\label{Hart Ex 1.9.}

In the context of the prompt, both sheaves on space $X$ take value in category of Abelian group. Let $\{U_i\}_{i\in I}$ be a covering of arbitrary open subset $U\subset X$ where $I$ is some index set. Assume $f=(s,t)\in \mathscr F\oplus\mathscr G(U)$ for some $s\in\mathscr F(U)$ and $t\in\mathscr G(U)$. 
\begin{align*}
    f\vert_{U_i}=\operatorname{Res}_{U,U_i}(f)=\operatorname{Res}_{U,U_i}((s,t))=(\operatorname{Res}_{U,U_i}(s),\operatorname{Res}_{U,U_i}(t))=0\in \mathscr F(U_i)\oplus\mathscr G(U_i).
\end{align*}
Then we must have $s\vert_{U_i}=0$ for all $i$, then $s\vert_U=0$ for $\mathscr F$ is a sheaf. Then we can conclude that $f\vert_{U}=(s\vert_U,t\vert_U)=0$.

For gluability, ...

\subsection{Warning}

Example \href{https://stacks.math.columbia.edu/tag/006Y}{006Y} on \cite{stacks-project}, which shows we cannot extend to taking infinitely many direct sum: it is a presheaf, but not sheaf in general.

\subsection{Hint}

See \cite{rotman2009introduction} Proposition 5.78. on Page 294.

\section{Exercise 1.14.}\label{Hart Chap 2 Ex 1.14.}

Because for any point $q\in X$ such that $s_q=0$, we can find a neighborhood $U_q$ of $q$ such that $\operatorname{Res}_{?U_q}s=0$. Hence for any point $q'\in U_q$, we have $q'\notin \operatorname{Supp} s$, which proves it's closed for its complement is open.

\subsection{Solution}

See \href{https://math.stackexchange.com/questions/2080018/prove-that-the-support-of-a-section-of-a-sheaf-is-a-closed-set}{POST}, \href{https://math.stackexchange.com/questions/2418007/the-support-of-a-sheaf-is-not-necessarily-closed#:~:text=Then%20we%20have%20Su,not%20closed%20in%20Zariski%20topology.}{POST}, \href{https://stacks.math.columbia.edu/tag/01AS}{Tag 01AS}.

\section{Exercise 1.15.}

See Rising Sea 2.3.C.

\section{Exercise 1.18.}

\subsection{Solution}

In fact, the sheaf on $Y$ could be weaker: it suffices to assume $\mathcal G$ to be a presheaf.

See Proposition 2.27 Chapter 2 of \cite{gortz2020algebraic}, or see \href{https://stacks.math.columbia.edu/tag/008C}{Tag 008C}.


\section{Exercise 1.19.}\label{Hart Chap 2 Ex 1.19.}

\textit{\textbf{Extending a Sheaf by Zero.} Let $X$ be a topological space, let $Z$ be a closed subset, let $i:Z\to X$ be the inclusion, let $U=X-Z$ be the complementary open subsets, and let $j:U\to X$ be its inclusion.}

\subsection{(a)}
\textit{Let $\mathscr F$ be a sheaf on $Z$. Show that the stalk $(i_{\ast}\mathscr F)_P$ of the direct image sheaf on $X$ is $\mathscr F_P$ if $P\in Z$, $0$ if $P\notin Z$. Hence we call $i_{\ast}\mathscr F$ the sheaf obtained by extending $\mathscr F$ be zero outside $Z$. By abuse of notation we will sometimes write $\mathscr F$ instead of $i_{\ast}\mathscr F$, and say \enquote{consider $\mathscr F$ as a sheaf on $X$,} when we mean \enquote{consider $i_{\ast}\mathscr F$.}}
\begin{proof}
    For $P\in Z$, we have 
    \begin{align*}
        (i_{\ast}\mathscr F)_P =&~ \underset{P\in W\subset X,~ W ~\text{open}}{\Colim} (i_{\ast}\mathscr F)(W)\\
        =&~ \underset{P\in W\subset Z\subset X,~ W ~\text{open}}{\Colim} (i_{\ast}\mathscr F)(W)\\
        =&~ \underset{P\in W\subset Z\subset X,~ W ~\text{open}}{\Colim} \mathscr F(i^{-1}(W))\\
        =&~ \underset{P\in W\subset Z\subset X,~ W ~\text{open}}{\Colim} \mathscr F(W)\\  
        =&~ \mathscr F_P
    \end{align*}
    For $P\notin Z$, we have 
    \begin{align*}
        (i_{\ast}\mathscr F)_P =&~ \underset{P\in W\subset X,~ W ~\text{open}}{\Colim} (i_{\ast}\mathscr F)(W)\\
        =&~ \underset{P\in W\subset X,~ W ~\text{open},~ W\cap Z=\emptyset }{\Colim} (i_{\ast}\mathscr F)(W)\\
        =&~ \underset{P\in W\subset X,~ W ~\text{open},~ W\cap Z=\emptyset }{\Colim} \mathscr F(i^{-1}(W))\\ 
        =&~ \underset{P\in W\subset X,~ W ~\text{open},~ W\cap Z=\emptyset }{\Colim} \mathscr F(\emptyset)\\  
        =&~ 0.
    \end{align*}
    In summary, we know 
    \begin{align*}
        (i_{\ast}\mathscr F)_P = \begin{cases}
            \mathscr F_P &~ \text{ when } P\in Z\\
            0 &~ \text{ when } P\notin Z
        \end{cases}.
    \end{align*}
\end{proof}

\subsection{(b)}
\textit{Now let $\mathscr F$ be a sheaf on $U$. Let $j_{!}(\mathscr F)$ be the sheaf on $X$ associate to the pre-sheaf $V\mapsto \mathscr F(V)$ if $V\subset U$, $V\mapsto 0$ otherwise. Show that the stalk $(j_{!}(\mathscr F))_P$ is equal to $\mathscr F_P$ if $P\in U$, $0$ if $P\notin U$, and show that $j_{!}\mathscr F$ is the only sheaf on $X$ which has this property, and whose restriction to $U$ is $\mathscr F$. We call $j_!\mathscr F$ the sheaf obtained by \textbf{extending} $\mathscr F$ \textbf{by zero} outside $U$. }

\begin{proof}
    According to the definition, the sheaf $j_!\mathscr F$ is the sheafification of presheaf defined on $V\subset X$ by 
    \begin{align*}
        (j_!\mathscr F)^0 (V)=\begin{cases}
            \mathscr F(V) ~\text{ when } V\subset U\\
            0 ~\text{ otherwise}
        \end{cases}.
    \end{align*}
    Now we compute, for $P\in U$, 
    \begin{align*}
        (j_!\mathscr F)_P = (j_!\mathscr F)^0_P =&~ \underset{P\in W\subset X,~ W \text{ open in }X}{\Colim}~ (j_!\mathscr F)^0(W)\\
        =&~ \underset{P\in W\subset U\subset X,~ W \text{ open in }X}{\Colim}~ (j_!\mathscr F)^0(W)\\
        =&~ \underset{P\in W\subset U\subset X,~ W \text{ open in }X}{\Colim}~ \mathscr F(W)\\ 
        =&~ \mathscr F_P.
    \end{align*}
    When $P\notin U$, we have 
    \begin{align*}
        (j_!\mathscr F)_P = (j_!\mathscr F)^0_P =&~ \underset{P\in W\subset X,~ W \text{ open in }X}{\Colim}~ (j_!\mathscr F)^0(W)\\
        =&~ \underset{P\in W\subset U\subset X,~ W \text{ open in }X}{\Colim}~ (j_!\mathscr F)^0(W)\\
        =&~ \underset{P\in W\subset U\subset X,~ W \text{ open in }X, W\not\subset U }{\Colim}~ \mathscr F(W)\\ 
        =&~ \underset{P\in W\subset U\subset X,~ W \text{ open in }X, W\not\subset U }{\Colim}~ \mathscr 0\\  
        =&~ 0.
    \end{align*}
    In summary, we have 
    \begin{align*}
        (j_!(\mathscr F))_P =\begin{cases}
            \mathscr F_P &~ \text{ when } P\in U\\
            0 &~ \text{ when }P\notin U
        \end{cases}.
    \end{align*}
    Suppose we have another sheaf $\mathscr G$ on $X$ such that $\mathscr G_P=\mathscr F_P$ if $P\in U$ and $\mathscr G_P=0$ if $P\notin U$, then we necessarily have $j_!\mathscr F=\mathscr G$ by Proposition 1.1. Chap 2 \cite{hartshorne2013algebraic}. 
    Furthermore, we notice for any $P\in U$,
    \begin{align*}
        ((j_!\mathscr F)\vert_U)_P =&~ \mathscr F_P ~\Rightarrow~ (j_!\mathscr F)\vert_U =\mathscr F.
    \end{align*}
\end{proof}

\subsection{(c)}
\textit{Now let $\mathscr F$ be a sheaf on $X$. Show that there is an exact sequence of sheaves on $X$, \[0\to j_!(\mathscr F\vert_U)\to \mathscr F\to i_{\ast}(\mathscr F\vert_Z)\to 0.\]}

\begin{proof}
We need to define map from $j_!(\mathscr F\vert_U)\to \mathscr F$, it suffices to define morphism of (pre-sheaves) from $\mathscr G:=j_!(\mathscr F\vert_U)^0$ which denote the presheaf defined in part (b). For any $W\subset X$, we define $\psi$
\begin{align*}
    \psi(W):\mathscr G(W) &\to \mathscr F(W) = \begin{cases}
        0 &~ \text{ when } W\not\subset U\\
        \operatorname{id}_{\mathscr F(W)} &~ \text{ when } W\subset R
    \end{cases}.
\end{align*} Restriction map from any $\mathscr G(W)$ where $W\not\subset U$ is $0$ given $\mathscr G(W)=0$. Then we can check for $W_1\subset W_2\subset X$, the following diagram is commutative

% https://q.uiver.app/#q=WzAsNCxbMCwwLCJcXG1hdGhzY3IgRyhXXzIpIl0sWzEsMCwiXFxtYXRoc2NyIEYoV18yKSJdLFswLDEsIlxcbWF0aHNjciBHKFdfMSkiXSxbMSwxLCJcXG1hdGhzY3IgRihXXzEpIl0sWzAsMl0sWzAsMV0sWzEsM10sWzIsM11d
\[\begin{tikzcd}
	{\mathscr G(W_2)} & {\mathscr F(W_2)} \\
	{\mathscr G(W_1)} & {\mathscr F(W_1)}
	\arrow[from=1-1, to=1-2]
	\arrow[from=1-1, to=2-1]
	\arrow[from=1-2, to=2-2]
	\arrow[from=2-1, to=2-2]
\end{tikzcd}\]

in three cases respective: i) when $W_1,W_2\not\subset U$, ii) when $W_2\not\subset U$ and $W_1\subset U$, iii) $W_1,W_2\subset U$.

By universal property of sheafification, given $\mathscr F$ is a sheaf, there exists a morphism of sheaves \[j_!(\mathscr F\vert_U)\to\mathscr F.\]

For any $W\subset X$, we define $\phi$ 
\begin{align*}
    \phi(W):\mathscr F(W) &\to i_{\ast}(\mathscr F\vert_Z)(W)=\mathscr F\vert_Z(i^{-1}(W))=\mathscr F(W\cap Z) \\
    x &\mapsto x\vert_{W\cap Z}
\end{align*}by restriction. Clearly it's compatible with the restriction maps hence $\phi$ is indeed a morphism of sheaves
\[\mathscr F\to i_{\ast}(\mathscr F\vert_Z).\] 

With these two morphisms defined, it suffices to check exactness at each term. 
For $P\in X=U\amalg Z$, we firstly consider $x\in U$ where the following sequence is exact 
% https://q.uiver.app/#q=WzAsNSxbMCwwLCIwIl0sWzEsMCwiKGpfIShcXG1hdGhzY3IgRlxcdmVydF9VKSlfUD1cXG1hdGhzY3IgRl9QIl0sWzIsMCwiXFxtYXRoc2NyIEZfUCJdLFszLDAsIihpX3tcXGFzdH0oXFxtYXRoc2NyIEZcXHZlcnRfWikpX1A9MCJdLFs0LDAsIjAiXSxbMCwxXSxbMSwyXSxbMiwzXSxbMyw0XV0=
\[\begin{tikzcd}
	0 & {(j_!(\mathscr F\vert_U))_P=\mathscr F_P} & {\mathscr F_P} & {(i_{\ast}(\mathscr F\vert_Z))_P=0} & 0
	\arrow[from=1-1, to=1-2]
	\arrow[from=1-2, to=1-3]
	\arrow[from=1-3, to=1-4]
	\arrow[from=1-4, to=1-5]
\end{tikzcd}\]

For $P\in Z$, we have exact sequence again as follows.
% https://q.uiver.app/#q=WzAsNSxbMCwwLCIwIl0sWzEsMCwiKGpfIShcXG1hdGhzY3IgRlxcdmVydF9VKSlfUD0wIl0sWzIsMCwiXFxtYXRoc2NyIEZfUCJdLFszLDAsIihpX3tcXGFzdH0oXFxtYXRoc2NyIEZcXHZlcnRfWikpX1A9XFxtYXRoc2NyIEZfUCJdLFs0LDAsIjAiXSxbMCwxXSxbMSwyXSxbMiwzXSxbMyw0XV0=
\[\begin{tikzcd}
	0 & {(j_!(\mathscr F\vert_U))_P=0} & {\mathscr F_P} & {(i_{\ast}(\mathscr F\vert_Z))_P=\mathscr F_P} & 0
	\arrow[from=1-1, to=1-2]
	\arrow[from=1-2, to=1-3]
	\arrow[from=1-3, to=1-4]
	\arrow[from=1-4, to=1-5]
\end{tikzcd}\]
Therefore we know $0\to (j_!(\mathscr F\vert_U))_P\to \mathscr F_P\to (i_{\ast}(\mathscr F\vert_Z))_P\to 0$ for every $P\in X$, which proves 
\[0\to j_!(\mathscr F\vert_U)\to \mathscr F\to i_{\ast}(\mathscr F\vert_Z)\to 0\] is exact.
\end{proof}

\subsection{Remarks}

The sheaf $j_{!}\mathscr F$ is defined as sheafification of the presheaf that we denote as $\mathscr G$, i.e. $j_{!}\mathscr F=\mathscr G^{\sharp}$. We'll demonstrate that it's crutial to sheafify. We define the presheaf $\mathscr G$ on $U\subset X$ by 
\begin{align*}
    \mathscr G(V)=\begin{cases}
        \mathscr F(V) & \text{ for } V\subset U \\
        0 & \text{ for } V\not\subseteq U.
    \end{cases}
\end{align*}

Let $X=\mathbb R$. 
Now we fix an open subset $W\subset X$ with $\{W_i\}_{i\in I}$ being an open cover for some index set $I$. We define 
\[\mathscr F(U):=\{\text{ continuous function from } U\to \mathbb R~\}\]

\begin{itemize}
    \item It satisfies Identity Axiom: if we're given an element $s\in \mathscr G(W)$ such that $s\mid_{W_i}=0$, then $s=0$ as a function.
    \item Let the open subset $W\subset X$ endowed with an open covering $\{W_1,W_2\}$ such that $W_1\cap W_2\not\subset U$. Then we can pick to different non-zero constant function $C_1
    in\mathscr G(W_1),C_2\in\mathscr G(W_2)$. They agree on the overlap for $C_1\vert_{W_1\cap W_2}=0=C_2\vert_{W_1\cap W_2}$. But there cannot exist a continuous function on... 
\end{itemize}\todo{I didn't find a counterexample according to the hint on Rising Sea}

\subsection{Comments}

See this \href{https://math.stackexchange.com/questions/1682362/extension-by-zero-not-quasi-coherent}{POST}, and \href{https://stacks.math.columbia.edu/tag/03S2}{Tag 03S2}.

\subsection{A Counterexample}

\textit{Let $X=\{u_1,u_2\}$ be a topological space endowed with discrete topology. The inclusion map is $j:\{u_1\}\to X$. Assume we have a sheaf $\mathscr F$ on $\{u_2\}$ such that $\mathscr F(\{u_2\})\neq 0$. Show $j_{!}^{\text{pre}}\mathscr F$ is not a sheaf.}

\begin{proof}
    We denote $\mathscr G:=j_{!}^{\text{pre}}\mathscr F$.
    In general, the pre-sheaf $\mathscr G$ satisfies the Identity Axiom. However, we can pick a non-zero section $s_1\in\mathscr G(\{u_1\})$ and a section $s_2\in\mathscr G(\{u_2\})=0$. 
    Notice that 
    \[s_1\vert_{\{u_1\}\cap\{u_2\}}=s_1\vert_{\emptyset}=0=\cdots=s_2\vert_{\{u_1\}\cap\{u_2\}}.\] Hence if we assume $\mathscr G$ is a sheaf then there must exist a global section $s_3\in \mathscr G(X)=0$ such that 
    \[0=s_3\vert_{\{u_1\}}=s_1,\] contradiction.
\end{proof}

\subsection{References}

See Rising Sea \cite{RaviRisingSea} Exercise 23.4.F.

\section{Exercise 1.22.}

\subsection{Hint}
See Stacks Project \href{https://stacks.math.columbia.edu/tag/00AK}{Glueing Sheaves}, in which condition (2) is called \textit{glueing data}. 

According to the following \href{https://stacks.math.columbia.edu/tag/00AL}{Tag 00AL}, there exists (not necessarily unique) a sheaf $\mathscr F$ on $X$ such that \dots

It remains to use (1) in the prompt to verify such a sheaf is unique.

See a post \href{https://math.stackexchange.com/questions/455706/gluing-sheaves-can-we-realize-mathcalfw-as-some-kind-of-limit}{HERE}.