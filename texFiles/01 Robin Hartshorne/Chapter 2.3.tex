\section{Example 3.0.1.}\label{Hart Chap 2 Example 3.0.1.}

See Page 2 of this \href{https://ocw.mit.edu/courses/18-726-algebraic-geometry-spring-2009/3723a99e97b581828fd782b9ffd83921_MIT18_726s09_lec11_more_schemes.pdf}{NOTE}.

\section{Caution 3.1.1.}

It's finitely many of Noetherian topological spaces (in the affine case), which is again Noetherian. See \href{https://math.stackexchange.com/questions/3388747/on-the-definition-of-noetherian-scheme}{POST}.

\section{Proposition 3.2.}

\section{Definition: Base Extension}

\textit{If $S''\to S'\to S$ are two morphisms, then $(X\times_S S')\times _{S'}S''\cong X\times_{S}S''$.}
\begin{proof}
By universal property of $X\times_S S''$ we have a unique map $(X\times_S S')\times _{S'}S''\to X\times_{S}S''$. Furthermore, universal property of $X\times_S S'$ give rise to $X\times_S S''\to X\times_S S'$. And this enable us to use universal property of $(X\times_S S')\times_{S'}S''$ to produce a unique map backwards, which proves the isomorphism.
\end{proof}

\section{Definition: Stable under Base Extension}

\section{Example 3.3.1.}

Recall Example 3.0.1 \Cref{Hart Chap 2 Example 3.0.1.}, we know $\Spec k[x,y,t]/(ty-x^2)$ is an integral scheme given the global section ring is an integral domain.
For $k[x,y,t]/(ty-x^2)$ is an integral domain, see this \href{https://math.stackexchange.com/questions/3320367/mathbbcx-y-z-xy-z2-is-not-a-field}{POST}. 

\section{Exercise 3.6.}

\begin{proof}
Given $(X,\mathcal O)$ an integral scheme, then there exists a unique generic point $\xi$ by Exercise 2.9 in \Cref{Hart Chap 2 Ex 2.9.}. Now we compute the local ring at $\xi$. Pick an affine open $\Spec A\subset X$ which contains $\xi$, then we know the closure of $\xi$ in $\Spec A$ must be whole $\Spec A$. While $X$ is integral, section $A$ is an integral domain, hence $\xi$ corresponds to the zero ideal in $A$.
\begin{align*}
    \mathcal O_{\xi} =&~ \underset{U \ni \xi,~ U \text{ open in } X}{\Colim} \mathcal O(U)\\
    =&~ \underset{U \ni \xi,~ U \text{ open in } \Spec A}{\Colim} \mathcal O\vert_{\Spec A}(U)\\ 
    =&~ A_{0}=\Frac A,
\end{align*} for which we define as the \textit{function field} $K(X)$.

\end{proof}

\subsection{Hint}

See this \href{https://math.stackexchange.com/questions/218767/relation-of-function-field-of-a-scheme-to-the-local-ring-of-its-prime-divisor}{POST}.

\section{Exercise 3.7.}

\textit{}