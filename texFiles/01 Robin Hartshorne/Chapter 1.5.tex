\section{Exercise 5.1.}

\subsection{(a)}

According to this \href{https://www.desmos.com/calculator/08sanf19br}{picture}, (a) is a tacnode. 

%Denote $f_1:=x^2-x^4-y^4$, . Hence it's an integral domain, then we know $\langle f_1\rangle$ is a prime ideal. So we can choose generator $f_1$ for the affine variety
Denote $f_1:=x^2-x^4-y^4$, which is irreducible in UFD $k[x,y]$ hence it's prime. We can then compute the ideal defined by this affine variety 
\[I(Z(\langle f_1\rangle))=\sqrt{\langle f_1\rangle}=\langle f_1\rangle.\]
The dimension of the affine variety $Z(\langle f_1\rangle)$ is 
\begin{align*}
    \operatorname{dim} Z(\langle f_1\rangle) &= \operatorname{dim} k[x,y]/\sqrt{\langle f_1\rangle}=\operatorname{dim} k[x,y]-\operatorname{height} \langle f_1\rangle
\end{align*}
By Krull's Hauptidealsatz we know the minimal prime ideal $\mathfrak p$ that contains $\langle f_1\rangle$ has height exactly $1$. While $\langle f_1\rangle$ is a prime, we know it must be height of $1$. Then by Theorem 4.7 in the notes, we know the Jacobian matrix at a singular point $P$ cannot have rank $2-1=1$. While the matrix is $1\times 2$, it follows that the matrix can only have dimension $0$.

So we have to compute the Jacobian matrix of the above affine variety at some point $P\in \mathbb A^2$ on the affine variety. We choose $f_1$ itself as generators for the ideal of the affine variety and compute the Jacobian matrix
\begin{align*}
    J(P) &= \begin{pmatrix}
        \frac{\partial f_1}{\partial x}(P) & \frac{\partial f_1}{\partial y}(P)
    \end{pmatrix}=
            \begin{pmatrix}
                2x-4x^3 (P) & -4y^3 (P)
            \end{pmatrix}.
\end{align*}
Equivalently, we must havve $2x-4x^3 (P)=0$ and $-4y^3 (P)=0$.
Solving the equation, notice that $P$ must lies on the tacnode, we'll get $P=(0,0)$ is the only singular point.

\subsection{(b)}

According to this \href{https://www.desmos.com/calculator/c298mxdjtn}{picture}, (b) is a node.

Denote $f_2=xy-x^6-y^6$. Similarly, we choose $f_2$ itself as the generator for the affine variety it defined.
Again, we have to compute the Jacobian matrix of $Z(f_2)$ at $P=(0,0)$. 
\begin{align*}
    J(P) &= \begin{pmatrix}
        \frac{\partial f_2}{\partial x}(P) & \frac{\partial f_2}{\partial y}(P)
            \end{pmatrix}
        = \begin{pmatrix}
            y-6x^5 (P) & x-6y^5 (P)
        \end{pmatrix} 
        = \begin{pmatrix}
            0 & 0
        \end{pmatrix},
\end{align*}which as rank $0$. Solving the equations $y-6x^5(P)=0$ and $x-6y^5(P)=0$ will implies that $P=(0,0)$.

\subsection{(c)}

See this \href{https://www.desmos.com/calculator/orh1nqbgng}{picture}, then we know (c) is a cusp.
Denote $f_3=x^3-y^2-x^4-y^4$.
We just need to check the Jacobian matrix 
\begin{align*}
    J(P) &= \begin{pmatrix}
        \frac{\partial f_2}{\partial x}(P) & \frac{\partial f_2}{\partial y}(P)
            \end{pmatrix}
        = \begin{pmatrix}
            3x^2-4x^3(P) & -2y-4y^3 (P)
        \end{pmatrix}.
\end{align*}
Solving the equations for points on cusp will forces $P=(0,0)$.

\subsection{(d)}

See this \href{https://www.desmos.com/calculator/b7gltpeepv}{picture}, then we know (d) is the triple point. And we denote $f_4=x^2y+xy^2-x^4-y^4$.
Compute the Jacobian matrix gives us
\begin{align*}
    J(P) &= \begin{pmatrix}
        2xy+y^2-4x^3 (P) & x^2+2xy-4y^3 (P)
    \end{pmatrix}.
\end{align*}
Solving the equations $2xy+y^2-4x^3 (P)=0 $ and $x^2+2xy-4y^3 (P)=0$ will give us $P=(0,0)$.

\subsection{}

See a post \href{https://math.stackexchange.com/questions/1728013/is-the-affine-curve-y2-x4y4-in-mathbb-a2-singular}{HERE}.

See REB's solution \href{https://math.berkeley.edu/~reb/courses/256A/1.5.pdf}{HERE}.

See a post on irreducibility of polynomial over $\mathbb C$ \href{https://math.stackexchange.com/questions/1335827/showing-a-polynomial-is-irreducible-over-mathbbcx-y}{HERE}.

\section{Exercise 5.3.}

\textit{\textbf{Multiplicities.} Let $Y\subset \mathbb A^2$ be a curve defined by the equation $f(x,y)=0$. Let $P=(a,b)$ be a point of $\mathbb A^2$. Make a linear change of coordinates so that $P$ becomes the point $(0,0)$. Then write $f$ as a sum $f=f_0+f_1+\cdots+f_d$, where $f_i$ is a homogeneous polynomial of degree $i$ in $x$ and $y$. Then we define the \textbf{multiplicity} of $P$ on $Y$, denoted $\mu_P(Y)$, to be the least $r$ such that $f_r\neq 0$. (Note that $P\in Y$ $\Leftrightarrow$ $\mu_P(Y)>0$.) The linear factors of $f_r$ are called the \textbf{tangent directions} at $P$.}\\\\

Notice that $P=(0,0)$ and 
\[P\in Y ~\Leftrightarrow~ f(P)=0 ~\Leftrightarrow~ f_0=0 ~\Leftrightarrow~ \mu_P(Y)>0.\]

\subsection{(a)}
\begin{proof}
    Notice that $\mu_P(Y)=1$ is equivalent to say that $f_1=ax+by\neq 0$ for some $a,b\in k$. Hence either $a$ or $b$ is non-zero. 
    Now we compute the Jacobian matrix at $P$. 
    \begin{align*}
        J(P)=&\begin{pmatrix}
                \frac{\partial f}{\partial x}(P) & \frac{\partial f}{\partial y}(P)
            \end{pmatrix}\\
            =&\begin{pmatrix}
                    0+\frac{\partial f_1}{\partial x}(P)+\frac{\partial f_2}{\partial x}(P)+\cdots & 0+\frac{\partial f_1}{\partial y} (P)+\frac{\partial f_2}{\partial y}(P)+\cdots
            \end{pmatrix}\\
            =&\begin{pmatrix}
                \frac{\partial f_1}{\partial x}(P) & \frac{\partial f_1}{\partial y}(P)
            \end{pmatrix}=\begin{pmatrix}
                a & b
            \end{pmatrix}.
    \end{align*}
    This $2\times 1$-matrix as dimension $1$ exactly because either $a$ or $b$ is nonzero. 

    If we can \todo{could we assume curve $f$ is irreducible? \ding{42} See \cite{hartshorne2013algebraic} Example 1.4.2. in Chapter 1 on Page 4} assume $\operatorname{dim}Y=1$, then it follows that $n-r=2-1$ equals to the rank of the Jacobian matrix, which proves that $P$ is non-singular on $Y$. The converse direction is similar. 

\end{proof}\todo{Verified \href{https://math.berkeley.edu/~reb/courses/256A/1.5.pdf}{HERE}}

\subsection{(b)}

See solution \href{https://math.berkeley.edu/~reb/courses/256A/1.5.pdf}{HERE}.

\section{Exercise 5.6.}
\textit{\textbf{Blowing Up Curve Singularities.}}
\subsection{(a)}
\textit{Let $Y$ be the cusp or node of (Ex. 5.1). Show that the curve $\widetilde{Y}$. obtained by
blowing up $Y$ at $O = (0,0)$ is nonsingular (cf. (4.9.1) and (Ex. 4.10)).}

\begin{proof}
    Let $Y$ be the node curve, so $Y=I(xy-x^6-y^6)$. Let $x,y$ be coordinate of $\mathbb A^2$ and let $u,v$ be coordinates for $\mathbb P^1$. For $\mathbb A^2_{x,y}\times \mathbb P^1_{u,v}$, we know the blowing-up of $\mathbb A^2$ at $O$ is 
    \begin{align*}
        \operatorname{Bl}_{O}\mathbb A^2=Z(xv-yu)\subset \mathbb A^2\times \mathbb P^1.
    \end{align*}
    And we denote the projection as $\varphi:\operatorname{Bl}_{O}\mathbb A^2\to \mathbb A^2$.
    Now we're going to compute strict transform of $Y$ 
    \begin{align*}
        \widetilde{Y}=\overline{\varphi^{-1}(Y\setminus \{O\})}.
    \end{align*}
    For any point $a_0\in (x,y,u:v)\in \widetilde Y$, we know $\varphi (a_0)=(x,y)\in Y$. Hence we know at least 
    \[\widetilde Y\subset Z(xv-yu,xy-x^6-y^6).\]

    Now we try to restrict ourselves to one affine cover $U_u$ of $\mathbb P^1$, i.e. let $u=1$. Then 
    \begin{align*}
        Z(xv-yu,~xy-x^6-y^6)\cap U_u &=Z(xv-y,~ xy-x^6-y^6)\\
        &=Z(x^2(v-x^4-x^4v^6))\subset \mathbb A^3.
    \end{align*}
    Here $Z(x^2)$ is the exceptional set. And $\widetilde{Y}\cap U_u=Z(v-x^4-x^4v^6)$. Take $f_1=v-x^4-x^4v^6$ and compute the Jacobian matrix at $P\in \mathbb A^2_{x,v}$
    \begin{align*}
        J(P) =& \begin{pmatrix}
                \partial f_1/\partial x (P) & \partial f_1/\partial v (P)
                \end{pmatrix}
            = \begin{pmatrix}
                -4x^3-4v^6x^3 (P) & 1-6x^4v^5 (P)
            \end{pmatrix}.
    \end{align*}
    Notice that there's no solution of $P$ for equations $-4x^3-4v^6x^3(P)=0$ and $1-6x^4v^5 (P)=0$. Therefore the matrix has rank exactly $1$ becuase the coefficient $k$ is a field. And by Krull's Hauptidealsatz, we know the dimension for the curve is $1$. Then apply Theorem 4.7. from the notes we know $2-1=1$ is exactly the rank of the Jacobian matrix. Hence there's no singular points on $\widetilde{Y}$.

    Similarly, we can check there's no singular points of $\widetilde{Y}$ on another affine cover $U_v$ where $v=1$. Hence we can conclude $\widetilde{Y}$ is non-singular.\todo{is it affine?}
\end{proof}

\subsection{(b)}
\textit{We define a \textbf{node} (also called \textbf{ordinary double point}) to be a double point (i.e., a point of multiplicity 2) of a plane curve with distinct tangent directions (Ex. 5.3). If $P$ is a node on a plane curve $Y$, show that $\varphi^{-1}(P)$ consists of two distinct nonsingular points on the blown-up curve $\widetilde{Y}$. We say that "blowing
up P resolves the singularity at P".}

\begin{proof}
    
\end{proof}

\subsection{(c)}

\textit{Let $P\in Y$ be the tacnode of (Ex. 5.1). If $\varphi:\widetilde Y\to Y$ is the blowing-up at $P$, show that $\varphi^{-1}(P)$ is a node. Using (b) we see that the tacnode can be resolved by two successive blowings-up.}

\subsection{(d)}

\textit{Let $Y$ be the plane curve $y^3=x^5$, which has a "higher order cusp" at $O$. Show that $O$ is a triple point; that blowing up $O$ gives rise to a double point (what kind?) and that one further blowing up resolves the singularity.}


\section{Exercise 5.12.}

\textit{\textbf{Quadric Hypersurfaces}. Assume $\operatorname{char} k\neq 2$, and let $f$ be a homogeneous polynomial of degree $2$ in $x_0,...,x_n$.}

\subsection{(a)}

\textit{Show that after a suitable linear change of variables, $f$ can be brought into the form $f=x_0^2+\cdots+x_r^2$ for some $0\leq r\leq n$. }

\begin{proof}
    Notice that it suffices to prove, after suitable linear transformation of variables, we can kill all terms $x_ix_j$ where $i\neq j$. Because then we know the polynomial will be $b_0x_0^2+\cdots+b_rx_r^2$, and we simply let $x_r\mapsto 1/\sqrt{b_r} x_r$ will yield the desired form. Denote our homogeneous polynomial $f$ as 
    \begin{align*}
        f=&a_{00}'x_0^2+a_{01}'x_0x_1+\cdots+a_{0n}'x_0x_n\\
        +&a_{10}'x_1x_0+\cdots+ \\
        +&a_{n0}'x_nx_0+\cdots+a_{nn}'x_nx_n\\
        =& \sum_{0\leq i\leq j\leq n} a_{ij}x_ix_j\\
        =&a_{00}x_0^2+a_{01}x_0x_1+a_{02}x_0x_2+\cdots+a_{nn}x_nx_n.
    \end{align*}
    Given $\operatorname{char} k\neq 2$, we know $1/2\neq 0$. We denote a symmetric $(n+1)\times (n+1)$-matrix with coefficients in $k$ by 
    \begin{align*}
        A=\begin{pmatrix}
            a_{00} & a_{01}/2 & a_{02}/2 & \cdots & a_{0n}/2\\
            a_{01}/2 & a_{11} & a_{12}/2 & \cdots & a_{1n}/2\\
            \vdots & \ddots  \\
            a_{0n}/2 & a_{1n}/2 & a_{2n}/2 & \cdots & a_{nn}.
        \end{pmatrix}
    \end{align*}.
    The matrix $A$ is symmetric and except the diagonal, every coefficient has an extra coefficient $1/2$. Let a vector be $\mathbf v=[x_0~ x_1~ \cdots x_n]$. The reason we introduce this matrix is becuase the following identity 
    \begin{align*}
        \mathbf v A \mathbf v^{t} = &a_{00}x_0x_0+1/2 a_{01}x_0x_1+\cdots 1/2 a_{0n}x_0x_n\\
        + &1/2a_{01}x_0x_1+a_{11}x_1x_1+1/2a_{12}x_1x_2+\cdots+1/2a_{2n}x_2x_n\\
        + &\cdots\\
        + &1/2a_{0n}x_0x_n+\cdots a_{nn}x_nx_n\\
        = &a_{00}x_0x_0+a_{01}x_0x_1+\cdots+a_{nn}x_nx_n=f.
    \end{align*}
    While $A$ is symmetric and over an algebraically closed field $k$, we can diagonalise it by some matrix $B$: 
    \begin{align*}
        BAB^{-1}=\begin{pmatrix}
            a_{00} & & \\
            & \ddots & \\
            & & a_{nn}
        \end{pmatrix}.
    \end{align*}
    Here matrix $B$ will provide information on linear change of variables. Also during linear change of variables, square terms can never be killed. So $r$ depends on number of square terms in the original polynomial $f$. \todo{?}
\end{proof}

\subsection{(b)}

\textit{Show that $f$ is irreducible if and only if $r\geq 2$.}

\begin{proof}
    Now we assume $f\in k[x_0,...,x_r]$ for some $0\leq r\leq n$.
    Suppose $f$ is reducible. So we can find a factorisation $f=f_1f_2$ for some non-unit polynomials $f_1,f_2\in k[x_0,...,x_r]$. This is equivalent to say that $f_1,f_2$ must be of homogeneous of degree $1$ given the coordinate ring $k[x_0,...,x_r]$ is an integral domain. 
    Given that $r\geq 2$, we can express the factorisation without loss of generality as 
    \begin{align*}
        f=f_1f_2=&(x_0+a_1x_1+\cdots+a_rx_r)(x_0+b_1x_1+\cdots+b_n
        rx_r)
    \end{align*}where all $a_i,b_i\in k\setminus \{0\}$ for $1\leq i\leq r$. 
    In order for the terms $x_ix_j$ where $i\neq j$ to be killed, we must have $a_ib_j+a_jb_i=0$ for all $0\leq i\leq j\leq r$. Here we assume $a_0=b_0=1$. Hence we immediately have $a_j=-b_j$ for all $1\leq j\leq r$ and the factorisation becomes 
    \begin{align*}
        (x_0+a_1x_1+\cdots+a_rx_r)(x_0-a_1x_1-\cdots-a_rx_r).
    \end{align*}
    And this means we can never kill the terms such as $x_1x_2$ for it has coefficent $2a_1a_2$. Then $f\neq f_1f_2$, contradiction. It follows that $f$ is irreducible.
    
\end{proof}

See a post \href{https://math.stackexchange.com/questions/39457/irreducibility-criteria-for-homogeneous-polynomials}{HERE}.