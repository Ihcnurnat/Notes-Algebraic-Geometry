\section{Definition: Morphism of sheaves of modules}

For a non-example... Notice that there are morphism of two modules that's not module morphism (i.e. does not respect action of the ring, but it's indeed a morphism between two Abelian groups). See \href{https://math.stackexchange.com/questions/1856824/is-a-group-homomorphism-a-module-homomorphism}{HERE}, in which they considered conjugation between to $\mathbb C$-module(over itself)
\[\overline{\cdot}:\mathbb C\to\mathbb C\]
and noticed that $\overline{i\cdot 1}\neq i\cdot \overline{1}$.

\section{Definition: Tensor Product of Sheaves}

See this \href{https://math.stackexchange.com/questions/1488296/tensor-product-of-sheaves-is-not-a-sheaf}{POST}.

\begin{comment}
\subsection{A Counterexample by ChatGPT}

Let \( X = \mathbb{R} \) be the real line, and consider the following covering:
\[ U_1 = (-\infty, 1) \quad \text{and} \quad U_2 = (-1, \infty) \]

We define two sheaves:
\begin{itemize}
    \item $\mathcal{F}$ is the sheaf of continuous real-valued functions on $X$.
    \item $\mathcal{G}$ is the sheaf of bounded real-valued functions on $X$.
\end{itemize}
Now, let's define the presheaf $\mathcal{H}$ on $X$ as the tensor product presheaf $\mathcal{F} \otimes \mathcal{G}$, where:


Consider the continuous, bounded functions \( f_1(x) = e^x \) defined on \( U_1 \) and \( f_2(x) = e^{-x} \) defined on \( U_2 \). These functions agree on the intersection \( U_1 \cap U_2 = \emptyset \).

However, if we attempt to glue these functions together to form a candidate section \( s \) over \( X \), we encounter a problem. Since \( f_1(x) \) and \( f_2(x) \) have different behaviors on the intervals \( U_1 \) and \( U_2 \), respectively, there is no continuous, bounded function that can extend both \( f_1(x) \) and \( f_2(x) \) over the entire real line \( X \).
\end{comment}


\section{Definition: Quasi-coherent Sheaves}

See 6.1.2. Theorem of Rising Sea. 

\section{Example 5.2.4.}

Consider a closed immersion $\iota:\Spec \mathbb F_{p}=Y\to \Spec \mathbb Z=X$. I'm confused by the notation $\mathscr O_{X\vert Y}$. But in fact we have 
\[\mathscr O_{X\vert Y}=\iota^{-1}(\mathscr O_{\Spec \mathbb Z})=(\mathscr O_{\Spec \mathbb Z})_{\mathfrak p}=\mathbb Z_{(p)}.\]
Then $\mathscr O_{X\vert Y}(Y)$ is not $\mathscr O_Y$-module. Because $\mathscr O_Y(Y)=\mathbb F_{p}$, while $\mathscr O_{X\vert Y}(Y)=...=\mathbb Z_{(p)}$. 
However, $\mathbb Z_{(p)}$ doesn't admit $\mathbb F_p$-module structure. Because for any non-zero element $x\in \mathbb Z_{(p)}$, we have 
$$0= px = x+...+x\in\mathbb Z_{(p)}, $$ which is impossible.