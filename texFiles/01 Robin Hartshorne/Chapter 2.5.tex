\section{Definition: Tensor Product of Sheaves}

See this \href{https://math.stackexchange.com/questions/1488296/tensor-product-of-sheaves-is-not-a-sheaf}{POST}.

\begin{comment}
\subsection{A Counterexample by ChatGPT}

Let \( X = \mathbb{R} \) be the real line, and consider the following covering:
\[ U_1 = (-\infty, 1) \quad \text{and} \quad U_2 = (-1, \infty) \]

We define two sheaves:
\begin{itemize}
    \item $\mathcal{F}$ is the sheaf of continuous real-valued functions on $X$.
    \item $\mathcal{G}$ is the sheaf of bounded real-valued functions on $X$.
\end{itemize}
Now, let's define the presheaf $\mathcal{H}$ on $X$ as the tensor product presheaf $\mathcal{F} \otimes \mathcal{G}$, where:


Consider the continuous, bounded functions \( f_1(x) = e^x \) defined on \( U_1 \) and \( f_2(x) = e^{-x} \) defined on \( U_2 \). These functions agree on the intersection \( U_1 \cap U_2 = \emptyset \).

However, if we attempt to glue these functions together to form a candidate section \( s \) over \( X \), we encounter a problem. Since \( f_1(x) \) and \( f_2(x) \) have different behaviors on the intervals \( U_1 \) and \( U_2 \), respectively, there is no continuous, bounded function that can extend both \( f_1(x) \) and \( f_2(x) \) over the entire real line \( X \).
\end{comment}